\documentclass[12pt]{article}
 
\usepackage[margin=1in]{geometry} 
\usepackage{amsmath,amsthm,amssymb}
 
\newcommand{\N}{\mathbb{N}}
\newcommand{\Z}{\mathbb{Z}}
 
\newenvironment{theorem}[2][Theorem]{\begin{trivlist}
\item[\hskip \labelsep {\bfseries #1}\hskip \labelsep {\bfseries #2.}]}{\end{trivlist}}
\newenvironment{exercise}[2][Exercise]{\begin{trivlist}
\item[\hskip \labelsep {\bfseries #1}\hskip \labelsep {\bfseries #2.}]}{\end{trivlist}}
\newenvironment{problem}[2][Problem]{\begin{trivlist}
\item[\hskip \labelsep {\bfseries #1}\hskip \labelsep {\bfseries #2.}]}{\end{trivlist}}
\newenvironment{question}[2][Question]{\begin{trivlist}
\item[\hskip \labelsep {\bfseries #1}\hskip \labelsep {\bfseries #2.}]}{\end{trivlist}}
\newenvironment{corollary}[2][Corollary]{\begin{trivlist}
\item[\hskip \labelsep {\bfseries #1}\hskip \labelsep {\bfseries #2.}]}{\end{trivlist}}
\theoremstyle{definition}
\newtheorem{definition}{Definition}[section]
\theoremstyle{definition}
\newtheorem{lemma}{Lemma}[section]
\theoremstyle{definition}
\newtheorem{exmp}{Example}[section]
\theoremstyle{definition}
\newtheorem{remark}{Remark}[section]
%-------------------------------------- 
\begin{document}

\title{Econ Problem Set 1}
\author{Matthew Brown\\ 
OSM Boot Camp 2018} %if necessary, replace with your course title
 
\maketitle
 
\begin{problem}{1}
\begin{itemize}
\item (i). The state variables are the future sequence of prices and the number of barrels that are available to the owner.
\item (ii). The control variable is the number of barrels sold.
\item (iii). The transition equation is 
$$ b_{t+1} = b_t - k_t $$
where $b_t$ = the number of barrels available at the start of period $t$ and $k_t$ = the number of barrels sold in period $t$. Define also that $p_t$ = the price in period $t$.
\item (iv). The sequence problem is:
$$ max_{\{k_t\}_{t=1}^\infty} \bigg\{ \sum_{t=1}^{\infty} \frac{1}{(1+r)^{t-1}}p_tk_t \bigg\} \text{ such that } b_0 = \sum_{t=1}^\infty k_t  $$
while the Bellman equation is:
$$ V(B) = max_{0 \leq k \leq B} \bigg\{ pk + \frac{1}{1+r}V(B') \bigg\} $$
where $B' = B-k$
\item (v). I have a few constraints. First of all, the lifetime budget constraint given above:
$$b_0 = \sum_{t=1}^\infty k_t$$ 
The $=$ is actually a $\geq$, but it is always optimal to sell more, assuming positive price. 
We also have the constraint that for all t, 
$$k_t \geq 0$$. Therefore, the Lagrangian looks like this:
\begin{align*}
\mathcal{L}(k_1, ..., k_n, ..., \lambda) = \sum_{t=1}^\infty \frac{1}{(1+r)^{t-1}}p_tk_t +\lambda(b_0 - \sum_{t=1}^\infty k_t)
\end{align*}
The first order conditions are:
\begin{equation}
\frac{1}{(1+r)^{t-1}}^{t-1}p_t = \lambda \text{ for all } t
\end{equation}
\begin{equation}
b_0 = \sum_{t=1}^\infty k_t
\end{equation}
Combining two periods of (1), \begin{align*}
\frac{1}{(1+r)^{t-1}}p_t = \frac{1}{(1+r)^{t}}p_{t+1} \\
\implies p_t = \frac{1}{(1+r)}p_{t+1}
\end{align*}
This is the Euler equation, which has a weird look to it because there is no $k_t$ choice term, but this is because the problem is effectively describing linear utility.
\item (vi). If $p_{t+1} = p_t$ for all $t$, it is optimal to sell all the oil in the first period, because there is no discount term on the first period sales. If $p_{t+1} > \frac{1}{1+r}$ for all $t$, it would never be optimal to sell the oil and the owner would continue to hold it. ***
\end{itemize}
\end{problem}

\begin{problem}{2}
\begin{itemize}
\item (i) The state variables are the capital stock $k_t$, and the size of the shock $z_t$. Income $y_t$ could also maybe be considered a state variable, but it is completely determined by these other two.
\item (ii) The control variables are consumption and investment, $(c_t)$ and $(i_t)$.
\item (iii) The Bellman equation is: 
\begin{equation}
V(k_t, z_t) = max_{c_t}  \big\{ E_0(c_t) + \beta E_{z_{t+1}|z_t}V(k_{t+1},z_{t+1}) \big\}
\end{equation}
where $k_{t+1} = (1 - \delta)k_t + (y_t - c_t) =(1-\delta)k_t + z_tk_t - c_t$ (law of motion). Also, $ E_{z_{t+1}|z_t} = E_{z_{t+1}}$  because the shocks are independently distributed. This should be enough to solve the model!
\item (iv) I've done this part of the exercise in the jupyter notebook titled "PS1Ex2 - Neoclassical Growth"
\end{itemize}
\end{problem}
\begin{problem}{3}
\begin{itemize}
\item (i) The bellman equation is identical to the bellman equation we used in Problem 2 (See equation (3)). The difference here is that we cannot simply ignore the difference between $E_{z_{t+1}|z_t}$ and $E_{z_t}$ - because it is an autoregressive process, the shocks are correlated.
\item (ii) I've done this in the jupyter notebook titled "PS1Ex3 - Neoclassical Growth with Serial Correlation"
\end{itemize}
\end{problem}
\begin{problem}{4}
\begin{itemize}
\item (i) The Bellman equation is: 
$$
V(w) = max_{d \in \{1, 0\}} V^d(w)
$$
where $w$ is the wage that is offered in the current period, $w'$ is the wage offered in the next period, $b$ is the value of unemployment benefits, $d = 1$ is the decision to work, $d =0$ is the decision not to work, and 
\begin{align*}
V^1(w) = \sum_{t=0}^\infty \beta^t w \\ V^0(w) = \beta E_{w' | w} (V(w')) 
\end{align*}
\item (ii) 
\end{itemize}
\end{problem}
\end{document}