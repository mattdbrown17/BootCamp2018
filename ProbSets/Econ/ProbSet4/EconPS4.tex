\documentclass[12pt]{article}
 
\usepackage[margin=1in]{geometry} 
\usepackage{amsmath,amsthm,amssymb}
 
\newcommand{\N}{\mathbb{N}}
\newcommand{\Z}{\mathbb{Z}}
 
\newenvironment{theorem}[2][Theorem]{\begin{trivlist}
\item[\hskip \labelsep {\bfseries #1}\hskip \labelsep {\bfseries #2.}]}{\end{trivlist}}
\newenvironment{exercise}[2][Exercise]{\begin{trivlist}
\item[\hskip \labelsep {\bfseries #1}\hskip \labelsep {\bfseries #2.}]}{\end{trivlist}}
\newenvironment{problem}[2][Problem]{\begin{trivlist}
\item[\hskip \labelsep {\bfseries #1}\hskip \labelsep {\bfseries #2.}]}{\end{trivlist}}
\newenvironment{question}[2][Question]{\begin{trivlist}
\item[\hskip \labelsep {\bfseries #1}\hskip \labelsep {\bfseries #2.}]}{\end{trivlist}}
\newenvironment{corollary}[2][Corollary]{\begin{trivlist}
\item[\hskip \labelsep {\bfseries #1}\hskip \labelsep {\bfseries #2.}]}{\end{trivlist}}
\theoremstyle{definition}
\newtheorem{definition}{Definition}[section]
\theoremstyle{definition}
\newtheorem{lemma}{Lemma}[section]
\theoremstyle{definition}
\newtheorem{exmp}{Example}[section]
\theoremstyle{definition}
\newtheorem{remark}{Remark}[section]
%-------------------------------------- 
\begin{document}

\title{Econ Problem Set 4}
\author{Matthew Brown\\ 
OSM Boot Camp 2018} %if necessary, replace with your course title
 
\maketitle
 
\begin{section}{DSGE}
\begin{problem}{1}
The first step is to substitute the form of the policy function ($K_{t+1} = e^{z_t}AK_{t}^\alpha$) into the euler equation. We see that on the left side:
\begin{equation}
\frac{1}{e^{z_t}K_t^\alpha - K_{t-1}} = \frac{1}{e^{z_t}K_t^\alpha(1- A)}
\end{equation}
And on the right side, we have
\begin{align}
\beta \mathbb{E} \Big( \frac{\alpha e^{z_{t+1}} K_{t+1}^{\alpha - 1}}{e^{z_t+1} K_{t+1}^\alpha - K_{t+2}} \Big) &= \beta \mathbb{E} \Big( \frac{\alpha e^{z_{t+1}} (e^{z_t}AK_{t}^\alpha)^{\alpha - 1}}{e^{z_{t+1}} (e^{z_t}AK_{t}^\alpha)^\alpha - (e^{z_t+1}A(e^{z_t}AK_{t}^\alpha))^\alpha} \Big) \\ 
&= \beta \mathbb{E} \Big( \frac{\alpha e^{z_{t+1}} (e^{z_t}AK_{t}^\alpha)^{\alpha - 1}}{e^{z_{t+1}}(e^{z_t}AK_{t}^{\alpha^2})(1 - A)} \Big) \\
&=  \beta \mathbb{E} \Big( \frac{\alpha}{ (e^{z_t}AK_{t}^\alpha)(1- A)}\Big)
\end{align}
Since we are in period $t$ and there is no uncertainty about anything inside the expected value, we can say
\begin{equation}
\frac{1}{e^{z_t}K_t^\alpha(1- A)} = \frac{\alpha \beta}{ (e^{z_t}AK_{t}^\alpha)(1- A)}
\end{equation}
Which is true if and only if $\alpha \beta = A$.
\end{problem}

\begin{problem}{2}
The characterizing equations are:
\begin{itemize}
\item \textbf{Household}
\begin{align}
w_t(1-\tau)\frac{1}{c_t}) = \frac{a}{1-l_t} \\
\frac{1}{c_t} = \beta \mathbb{E} \big[ \frac{1}{c_{t+1}}[(r_{t+1}-\delta)(1-\tau) +1] \big]
\end{align}

\item \textbf{Firm}
\begin{align}
R_t = \alpha e^{z_t} K^{\alpha - 1} L^{1-\alpha} \\
W_t = (1 - \alpha) e^{z_t} K^\alpha L^{-\alpha}
\end{align}
\item \textbf{Government}
\begin{equation}
\tau[w_tl_t(r_t - \delta)k_t] = T_t
\end{equation}
\item \textbf{Market Clearing and Price Equivalence}
\begin{align}
l_t &= L_t \\
k_t &= K_t \\
w_t &= W_t \\
r_t &= R_t 
\end{align}
We can't use the same tricks to solve for A, because the functional form from problem 1 does not yield the same nice cancellation.
\end{itemize}
\end{problem}

\begin{problem}{3}
Characterizing equations:
\begin{itemize}
\item \textbf{Household}
\begin{align}
w_t(1-\tau)c_t^{-\gamma} = \frac{a}{1-l_t} \\
c_t^{-\gamma} = \beta \mathbb{E} \big[ c_{t+1}^{-\gamma}[(r_{t+1}-\delta)(1-\tau) +1] \big]
\end{align}

\item \textbf{Firm}
\begin{align}
R_t = \alpha e^{z_t} K_t^{\alpha - 1} L_T^{1-\alpha} \\
W_t = (1 - \alpha) e^{z_t} K_t^\alpha L_t^{-\alpha}
\end{align}
\item \textbf{Government}
\begin{equation}
\tau[w_tl_t(r_t - \delta)k_t] = T_t
\end{equation}
\item \textbf{Market Clearing and Price Equivalence}
\begin{align}
l_t &= L_t \\
k_t &= K_t \\
w_t &= W_t \\
r_t &= R_t 
\end{align}
\end{itemize}
\end{problem}
\begin{problem}{4}
Characterizing Equations:
\begin{itemize}
\item \textbf{Household}
\begin{align}
w_t(1-\tau)c_t^{-\gamma} = -a(1-l_t)^{-\xi} \\
c_t^{-\gamma} = \beta \mathbb{E} \big[ c_{t+1}^{-\gamma}[(r_{t+1}-\delta)(1-\tau) +1] \big] 
\end{align}

\item \textbf{Firm}
\begin{align}
R_t = e^z_t [\alpha K^{\eta}_t +(1-\alpha)L^{\eta}_t]^{\frac{1}{\eta} - 1} \alpha K_t^{\eta -1} \\
W_t = e^z_t [\alpha K^{\eta}_t +(1-\alpha)L^{\eta}_t]^{\frac{1}{\eta} - 1} (1-\alpha) L_t^{\eta -1}
\end{align}
\item \textbf{Government}
\begin{equation}
\tau[w_tl_t(r_t - \delta)k_t] = T_t
\end{equation}
\item \textbf{Market Clearing and Price Equivalence}
\begin{align}
l_t &= L_t \\
k_t &= K_t \\
w_t &= W_t \\
r_t &= R_t 
\end{align}
\end{itemize}
\end{problem}

\begin{problem}{5} Characterizing Equations:
\begin{itemize}
\item \textbf{Household}
\begin{align}
c_t^{-\gamma} = \beta \mathbb{E} \big[ c_{t+1}^{-\gamma}[(r_{t+1}-\delta)(1-\tau) +1] \big]
\end{align}

\item \textbf{Firm}
\begin{align}
R_t = \alpha e^{\alpha z_t} K_t^{\alpha - 1} L_T^{1-\alpha} \\
W_t = (1 - \alpha) e^{\alpha z_t} K_t^\alpha L_t^{- \alpha}
\end{align}

\item \textbf{Government}
\begin{equation}
\tau[w_t(r_t - \delta)k_t] = T_t
\end{equation}
\item \textbf{Market Clearing and Price Equivalence}
\begin{align}
1 &= L_t \\
k_t &= K_t \\
w_t &= W_t \\
r_t &= R_t 
\end{align}

\end{itemize}
In steady state we get:
\begin{align}
c^{-\gamma} = \beta \big[ c^{-\gamma}[(r-\delta)(1-\tau)+1] \big] \\
r = \alpha k^{\alpha - 1} \\
w = (1 - \alpha)  k^\alpha \\
\tau[w(r - \delta)k] = T
\end{align}
Solving analytically, we see that 
\begin{align}
r = \frac{\frac{1}{\beta} - 1}{1 - \tau} + \delta = 7.287 \\
k = \Big( \frac{r}{\alpha} \Big)^{\frac{1}{\alpha - 1}} = .121 \\
w = (1 - \alpha) k^\alpha = 1.328 \\
Y = k^\alpha = .729 \\
I = \delta k = 2.213
\end{align}
The values of $r, w, k$, and $c$ are solved numerically in the jupyter notebook, and they confirm these numbers.
\end{problem}

\begin{problem}{6} Characterizing Equations (household from problem 4, firm from problem 5):
\begin{itemize}
\item \textbf{Household}
\begin{align}
w_t(1-\tau)c_t^{-\gamma} = -a(1-l_t)^{-\xi} \\
c_t^{-\gamma} = \beta \mathbb{E} \big[ c_{t+1}^{-\gamma}[(r_{t+1}-\delta)(1-\tau) +1] \big] 
\end{align}

\item \textbf{Firm}
\begin{align}
R_t = \alpha e^{\alpha z_t} K_t^{\alpha - 1} L_T^{1-\alpha} \\
W_t = (1 - \alpha) e^{\alpha z_t} K_t^\alpha L_t^{- \alpha}
\end{align}
\item \textbf{Government}
\begin{equation}
\tau[w_tl_t(r_t - \delta)k_t] = T_t
\end{equation}
\item \textbf{Market Clearing and Price Equivalence}
\begin{align}
l_t &= L_t \\
k_t &= K_t \\
w_t &= W_t \\
r_t &= R_t 
\end{align}
\end{itemize}
Steady state gives us:
\begin{align}
w(1 - \tau)c^{-\gamma} = - a (1 - l)^{-\xi} \\
c^{-\gamma} = \beta \big[c^{-\gamma} [(r - \delta)(1 - \tau) + 1]\big] \\
r = \alpha k^{\alpha - 1}l^{1-\alpha} \\
w = (1 - \alpha) k^\alpha l^{-\alpha} \\
\tau[wl(r - \delta)k] = T
\end{align}
\end{problem}
These are solved numerically in the Jupyter notebook.

\end{section}



\begin{section}{Linearization}

\begin{problem}{1}
I am very confused about how to log-linearize, so I didn't manage this or number 2.
\end{problem}

\begin{problem}{3}
I shifted everything up a period since this made more sense to me. The epsilons can disappear because they turn to 0 in expected value. Also every variable is supposed to have a tilde, but I might have missed some.
\begin{align*}
\mathbb{E} \{ F \tilde{X}_{t+2} + G \tilde{X}_{t+1} + H \tilde{X}_{t} + L \tilde{Z}_{t+2} + M \tilde{Z}_{t+1}\} = 0 \\
\mathbb{E} \{ F (P\tilde{X}_{t+1} + Q\tilde{Z}_{t+2}) + G (P\tilde{X}_{t} + Q\tilde{Z}_{t+1}) + H \tilde{X}_{t} + L (N \tilde{Z}_{t+1} + \epsilon_{t+1}) + M\tilde{Z}_{t+1} \} = 0 \\
\mathbb{E} \{((FP)\tilde{X}_{t+1} + FQ\tilde{Z}_{t+2} + (GP+H)\tilde{X}_t + (GQ + LN + M)\tilde{Z}_{t+1} \} = 0 \\
\mathbb{E} \{(FP)(P\tilde{X}_{t} + Q\tilde{Z}_{t+1}) + FQ(N \tilde{Z}_{t+1} + \epsilon_{t+1}) + (GP+H)\tilde{X}_t + (GQ + LN + M)\tilde{Z}_{t+1} = 0 \\ 
((FP+G)P + H)X_t + [(FQ + L)N + (FP + G)Q +M]\tilde{Z}_{t+1} = 0
\end{align*}

\end{problem}

\end{section}

\end{document}