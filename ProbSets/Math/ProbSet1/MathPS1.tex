\documentclass[12pt]{article}
 
\usepackage[margin=1in]{geometry} 
\usepackage{amsmath,amsthm,amssymb}
 
\newcommand{\N}{\mathbb{N}}
\newcommand{\Z}{\mathbb{Z}}
 
\newenvironment{theorem}[2][Theorem]{\begin{trivlist}
\item[\hskip \labelsep {\bfseries #1}\hskip \labelsep {\bfseries #2.}]}{\end{trivlist}}
\newenvironment{exercise}[2][Exercise]{\begin{trivlist}
\item[\hskip \labelsep {\bfseries #1}\hskip \labelsep {\bfseries #2.}]}{\end{trivlist}}
\newenvironment{problem}[2][Problem]{\begin{trivlist}
\item[\hskip \labelsep {\bfseries #1}\hskip \labelsep {\bfseries #2.}]}{\end{trivlist}}
\newenvironment{question}[2][Question]{\begin{trivlist}
\item[\hskip \labelsep {\bfseries #1}\hskip \labelsep {\bfseries #2.}]}{\end{trivlist}}
\newenvironment{corollary}[2][Corollary]{\begin{trivlist}
\item[\hskip \labelsep {\bfseries #1}\hskip \labelsep {\bfseries #2.}]}{\end{trivlist}}
\theoremstyle{definition}
\newtheorem{definition}{Definition}[section]
\theoremstyle{definition}
\newtheorem{lemma}{Lemma}[section]
\theoremstyle{definition}
\newtheorem{exmp}{Example}[section]
\theoremstyle{definition}
\newtheorem{remark}{Remark}[section]
%-------------------------------------- 
\begin{document}

\title{Math Problem Set 1}
\author{Matthew Brown\\ 
OSM Boot Camp 2018} %if necessary, replace with your course title
 
\maketitle
 
\begin{problem}{1.3}
There are three things to check:
\begin{itemize}
\item $G_1$ is neither an algebra nor a sigma-algebra
\item $G_2$ is only an algebra, but it is not a sigma-algebra as it is not closed under countable unions.
\item $G_3$ is both an algebra and a sigma-algebra. 
\end{itemize}
\end{problem}

\begin{problem}{1.7} We already showed that both of these sets are sigma-algebras. Obviously no sigma-algebra can be larger than $\mathcal{P}(X)$, since this is the largest collection of subsets of $X$. And since any sigma-algebra $S$ must contain $\emptyset$, $X \in S$, so $\{ \emptyset, X \} \subset S $.
\end{problem}

\begin{problem}{1.10}
\begin{proof}
I'll check the three axioms of sigma-algebras on $S = \bigcap_\alpha S_\alpha$:
\begin{itemize}
\item $\emptyset \in S_\alpha$ for each $\alpha$, so $\emptyset \in S$.
\item Let $E \in S$. Then $E \in S_\alpha$ for each $\alpha$. Since each $S_\alpha$ is a sigma-algebra, $E^c \in S_\alpha$ for each $\alpha$, which means that $E^c \in S$.
\item Very similar to the above. Let $\{E_i\}_{n==1}^{\infty} \in S$. Then ${E_i} \in S_\alpha$ for each $\alpha$ and each $i$, and $\cup_{i=1}^\infty E_i \in S_\alpha$ for each $\alpha$, so $\cup_{i=1}^\infty E_i \in S$.
\end{itemize}
\end{proof}
\end{problem}

\begin{problem}{1.17}
\begin{proof} There are two parts.
\begin{itemize}
\item \textit{Monotonicity:} \\
Let $A, B \in S, A \subset B$. Then we can decompose B as follows
\begin{align*}
B &= (B \cap A) \cup (B \cap A^c) \\
 &= A \cup (B \cap A^c)
\end{align*}
And since now we have written $B$ as a union of disjoint sets, we can say that $\mu(B) = \mu(A) + \mu(B \cap A^c) $ and, by the nonnegativaty of $\mu$, $\mu(B) \geq \mu(A)$.
\item \textit{Subadditivity:} \\
Let $\{ A_i \}_{i=1}^{\infty} \in S $. I will decompose $\cup_{i=1}^{\infty}A_i$ as follows: 
$$ \cup_{i=1}^{\infty}A_i = A_1 \cup (A_2 \cap A_1^c) \cup (A_3 \cap A_1^c \cap A_2^c) \cup ... \cup (A_i \cap A_1^c \cap ... \cap A_{i-1}^c) \cup ... $$
So that the $\cup_{i=1}^{\infty}A_i$ is written as a union of disjoint sets. Moreover, we see, for instance, that $\mu(A_2 \cap A_1^c) \leq \mu(A_2)$ by the monotonicty of $\mu$. Combining these two facts, we see that: 
\begin{align*}
\mu(\cup_{i=1}^{\infty}A_i) &= \mu(A_1) 
+ \mu(A_2 \cap A_1^c) + ... + \mu(A_i \cap A_1^c \cap ... \cap A_{i-1}^c) + ... \\
&\leq \mu(A_1)+\mu(A_2)+...+\mu(A_n)+...
\end{align*}
Which is what we wanted to show.
\end{itemize}
\end{proof}
\end{problem}

\begin{problem}{1.20}
\begin{proof}. This proof is adapted from Dr. Richard Timoney's lectures for a measure theory course at Trinity College. \\
Let $\{A_i\}_{i=1^{\infty}}$ be a decreasing sequence of measurable sets, as in the statement. Define $B_n = A_1 \setminus A_n$ for each $n \in \mathbb{N}$. See that for each $n$, $A_1 = B_n \cup A_n$ Because this is a disjoint union, we have $ \mu(A_1) = \mu(B_n) + \mu(A_n)$
and thus:
\begin{equation}
\mu(A_n) = \mu(A_1) - \mu(B_n)
\end{equation}
Now, $\{B_i\}_{i=1}^\infty$ is an increasing sequence of functions. Define $B = \cup_{i=1}^\infty B_i $, and from part (i) of the theorem, 
\begin{equation}
\mu(B) = lim_{n \to \infty} \mu(B_n) 
\end{equation} 
We'll use these two facts. \\
Now, note that \begin{align*}
A_1 \setminus \bigcap^{\infty}_i A_i = A_1 \cap \big(\bigcap^{\infty}_i A_i \big)^c  = \bigcup_{i=1}^\infty \big(A_1 \cap A_i^c\big) = \bigcup_{i=1}^\infty (A_1 \setminus A_i) = B
\end{align*}
Therefore, 
\begin{align*}
\mu(A_1) = \mu\big(A_1 \setminus \bigcap_{i=1}^\infty A_ii \big) + \mu \big(\bigcap_{i=1}^\infty A_i \big) = \mu (B) + \mu \big(\bigcap_{i=1}^\infty A_i \big)
\end{align*}
Therefore, as in equation (1), 
\begin{align*}
\mu\big(\bigcap_{i=1}^\infty A_i\big) = \mu(A_1) - \mu(B)
\end{align*} 
and by equation (2) we see that
\begin{align*}
\mu\big(\bigcap_{i=1}^\infty A_i\big) = \mu(A_1) - lim_{n \to \infty} \mu(B_n)
\end{align*}
And we can pass this limit outside and use equation (1), so that
\begin{equation}
\mu\big(\bigcap_{i=1}^\infty A_i\big) = lim_{n \to \infty} \big( \mu(A_1) - \mu(B_n) \big) = lim_{n \to \infty} \mu(A_n)
\end{equation}
And (3) was what we wanted to show.
\end{proof}
\end{problem}

\begin{problem}{2.10}
We know from the countable subadditivity of the outer measure that 
$$\mu^*(B) \leq \mu^*(B \cap E) + \mu^*(B \cap E^c) $$
So the $\geq$ in (*) will never actually be $>$, and it is equivalent to replace it with equality.
\end{problem}

\begin{problem}{2.14} Most of the work has been done for us elsewhere in the notes. I will show a brief intermediate step:
\begin{lemma}
The sigma algebra generated by $\mathcal{A}$ from Example 2.2 is the borel sigma algebra.
\end{lemma}
\begin{proof}
I know that there is an open set in $\mathcal(A)$, $\sigma(\mathcal{A})$ at least contains $\sigma(\mathcal{O})$. To show the other inclusion, recognize that intervals of the form $(a,b]$ and $(-\infty, a]$ are in $\sigma(\mathcal{O})$ since we can get the closed edge of the interval by taking a complement of an open interval (Eventually. - there are certainly missing details here, my apologies...) 
\end{proof}
The pre-measure $\nu$ as defined in example 2.2 is still a pre-measure on $\sigma(\mathcal{A}) = \sigma(\mathcal{O}).$ This means that we can apply the Caratheodory Extension Theorem 2.12 to see that $\sigma(\mathcal{O}) = \mathcal{M}$.
\end{problem}

\begin{problem}{3.1}
\begin{proof}
Let $X$ be a countable set in $\mathbb{R}$. Then I can ennumerate the elements of $X$ so that $\{x_1, ... \} = X$. Now, construct intervals $I_\epsilon^i$ for a given small $\epsilon$ and for each $i$ so that $I_\epsilon^i = (x_i - \frac{\epsilon}{2^i}, x_i + \frac{\epsilon}{2^i})$. Now for $$\mu(\bigcup_{i=1}^\infty I^i_\epsilon) \leq \sum_{i=1}^{\infty} \frac{2\epsilon}{2^i} = 2\epsilon$$
We see that even if we make $\epsilon$ arbitrarily small, $\bigcup_{i=1}^\infty I^i_\epsilon$ covers $X$, so the infimum of the measures of these covers is zero, which is the measure of $X$.
\end{proof}
\end{problem}

\begin{problem}{3.4}
This follows from the fact that the set of measurable sets is a sigma-algebra, and all these statements are equivalent.
\begin{proof}
I'll show that the sets being measurable are all equivalent statements. \\
$f^{-1}((-\infty, a))$ is measurable $\iff f^{-1}([a, \infty))$ is measurable (they are complements). I will now show that: 
$$ f^{-1}((-\infty, a)) \in \mathcal{M} \iff f^{-1}((-\infty, a]) \in \mathcal{M} $$
$\Rightarrow$: Suppose sets of the form $f^{-1}((-\infty, a]) \in \mathcal{M}$. Construct a sequence of sets $E_in = f^{-1}((-\infty, a - frac{1}{n}]) \in \mathcal{M} $. This countable union $\cup_{n=1}^\infty = f^{-1}((-\infty, a])$ is in $\mathcal{M}$ \\
$\Leftarrow$: Suppose sets of the form $f^{-1}((-\infty, a)) \in \mathcal{M}$. Then their complements, sets of the form $f^{-1} ([a, \infty))$ are also $in \mathcal{M}$. We can use a similar argument, employing sets of the form $f^{-1}([ a + \frac{1}{n}, \infty))$ to show that sets of the form $f^{-1}((a, \infty)) \in \mathcal{M}$. This shows that the complements of these sets, $f^{-1}((-\infty, a])$, are also elements of $\mathcal{M}$. \\
To conclude, see that $f^{-1}((a, \infty] \in \mathcal{M} \iff f^{-1}((-\infty, a]) \in \mathcal{M}$ because the sets are complements.
\end{proof} 
\end{problem}

\begin{problem}{3.7}
\begin{proof} I'll go item by item.
\begin{itemize}
\item $f+g$: $f+g: X \to \mathcal{M}$ can be written as $F(f , g): X \to \mathcal{M}$ where $F(x, y) = x + y$. This f is continuous, so it is measurable. 
\item $fg$: I can write $fg$ similarly as $F(f, g) = fg$ and use the same argument.
\item $min$: Define $\{f_n\}_{n=1}^\infty$ so that $f_1 = f, f_2 = f+2,... f_i = f+i$. Similarly define $\{g_n\}_{n=1}^\infty$ so that $g_1 = g, g_2 = g+2,... g_i = g+i$. $f$ and $g$ are the smallest elements of their respective sequences by construction. Now make a sequence $\{h_n\}_{n=1}^\infty$ where the odd terms are the $f$s and the even terms the $g$s. $inf_{n\in\mathbb{N}}\{h_n\}_{n=1}^\infty = min(f, g)$, and all the terms $h_i$ are lebesgue measurable so by (2) $min(f, g)$ is lebesgue measurable.
\item $max$: Just modify the above by making the sequences always be smaller that $f$ and $g$, e.g, $f_1 = f, f_2 = f - 2, ...$ and show that $max(f, g) = sup_{n\in\mathbb{N}}\{h_n\}_{n=1}^\infty$
\item \textit{Absolute value}: See that $|f| = max(f, -f)$, and therefore by what we've just shown it is lebesgue measurable.
\end{itemize}
\end{proof}
\end{problem}

\begin{problem}{3.14}
Recall the definition of uniform convergence: We want to show that 
$$ \forall \epsilon > 0, \exists N=N(\epsilon) \text{ such that } n \geq N \implies |f(x)-s_n(x)| < \epsilon, \forall x \in X$$
\begin{proof}
Suppose that $f(x) < M$. Fix $\epsilon > 0$. We construct intervals and simple functions just as we did in the proof of 3.13. Let $N_1 > M, N_1 \in \mathbb{N}$. Then $f(x) < N_1 $ for all x and $x \notin E_\infty^{N_1}$. We also see that there exists $N_2$ such that 
$$
N_2 > N_1 \text{ and } \frac{1}{2^{N_2}} < \epsilon
$$
Now, it follows that for $n>N_2$,  
$$
\forall x \in X, x \in E_i^n \text{ for some index } 0 \leq i \leq N_2, i \in \mathbb{N}
$$
Then $f(x) \in [\frac{i-1}{2^n}, \frac{i}{2^n}) $ and our simple function in this interval is $s_n(x) = \frac{i-1}{2^n} $ Recall now that we've chosen this same $N_2$ for ALL $x \in X$, and so $|f(x) - s_n(x)| < \frac{1}{2^n} < \frac{1}{2^{N^2}} < \epsilon $ implies uniform convergence.
\end{proof}
\end{problem}
\begin{problem}{4.13}
To show that $f \in \mathcal{L}^1(\mu, E)$, it suffices to show that $\int_Ef^+d\mu$ and $\int_Ef^-d\mu$ are both finite. The fact that $||f|| < M$ on $E$ implies that $\int_E|f|d\mu < M$. Thus, $\int_Ef^+d\mu$ and $\int_Ef^-d\mu$ are both finite, and we are done. (???)
\end{problem}
\begin{problem}{4.14}
I'll argue that the contrapositive is true.
\begin{proof}
Suppose that there exists a set $A$ such that $\mu(A)>0$ and $f(x) = \pm \infty$ Now, since $\mu(A) > 0$, either $B$ the set on which $f(x)=\infty$ or $C$ the set on which $f(x) =-\infty$ must have nonzero measure. So the proof splits into two cases. \\
\textit{Case 1}: $\mu(B) \neq 0$. Then 
$$\int_B fd\mu = \infty \implies \int_Bf^+d\mu = \infty $$ 
And we see then that 
$$\int_B f^+d\mu = \infty \implies \int_E f^+d\mu = \infty $$ 
because $B \subset E$ and $f^+$ is a positive function. And this means that $f \notin \mathcal{L}(E, \mu)$, which is what we wanted to show.\\
\textit{Case 2}: $\mu(C) \neq 0$. This case is exactly similar... We have
$$\int_C fd\mu = -\infty \implies \int_Cf^-d\mu = \infty $$ 
And we see then that 
$$\int_C f^-d\mu = \infty \implies \int_E f^-d\mu = \infty $$ 
because $C \subset E$ and $f^-$ is a positive function. And this means that $f \notin \mathcal{L}(E, \mu)$, which is what we wanted to show.
\end{proof}
\end{problem}

\begin{problem}{4.15}
\begin{proof}
Define $h = g - f.$ on $E$ Because $g \geq f, h(x) \geq 0$  and $\int_E hd\mu \geq 0 $. Consider now the following lemma.
\begin{lemma} (Linearity of Integral) \\
For any $f, g \in \mathcal{L}^1(E, \nu)$, $\alpha \in X$ 
\begin{align*}
&\int_E \alpha f f \mu = \alpha \int_E d \mu \\
&\int_E (f + g) d \mu = \int_E f d \mu + \int_E g d\mu
\end{align*}
I state this lemma without proof, since the property is somewhat intuitive (if this didn't hold, we would have quite a bad definition for an integral!). If a proof is desired, it can be found in the lecture notes to Richard Timoney's Lebesgue Integral course at Trinity College.
\end{lemma}
Using this fact, we see that 
$$\int_E g d\mu = \int_E f+h d\mu \geq \int_E f d\mu$$ 
Which is what we wanted to show
\end{proof}
\end{problem}

\begin{problem}{4.16}
\begin{proof}
Suppose that $f \in \mathcal{L}(E, \mu)$. Then by definition, $ \int_E f^+ d\mu < \infty, \int_E f^- d\mu < \infty$. Since $A \subset E$ and we're taking the integral of a positive function over a more restrictive domain, we can now say that $\int_A f^+ d\mu < \infty$ and $\int_A f^- d\mu < \infty$. And this means that $f \in \mathcal{L}(A, \mu)$.
\end{proof}
\end{problem}
\begin{problem}{4.21}
\begin{proof}
By the above theorem, $ \lambda( \cdot ) = \int_{\cdot} fd\mu $ is a measure on $\mathcal{M}$. So we know that it is countably additive, i.e, 
$$ \lambda(A) = \lambda(A \setminus B) + \lambda(A \cap B) = \lambda(A \setminus B) + \lambda(B) $$ and thus
$$ \int_A fd\mu = \int_{A \setminus B} fd\mu + \int_B fd\mu = \int_B fd\mu $$
(which is even more than what we wanted to show!)
\end{proof}
\end{problem}
\end{document}
