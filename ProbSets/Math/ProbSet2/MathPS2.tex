\documentclass[12pt]{article}
 
\usepackage[margin=1in]{geometry} 
\usepackage{amsmath,amsthm,amssymb}
 
\newcommand{\N}{\mathbb{N}}
\newcommand{\Z}{\mathbb{Z}}
 
\newenvironment{theorem}[2][Theorem]{\begin{trivlist}
\item[\hskip \labelsep {\bfseries #1}\hskip \labelsep {\bfseries #2.}]}{\end{trivlist}}
\newenvironment{exercise}[2][Exercise]{\begin{trivlist}
\item[\hskip \labelsep {\bfseries #1}\hskip \labelsep {\bfseries #2.}]}{\end{trivlist}}
\newenvironment{problem}[2][Problem]{\begin{trivlist}
\item[\hskip \labelsep {\bfseries #1}\hskip \labelsep {\bfseries #2.}]}{\end{trivlist}}
\newenvironment{question}[2][Question]{\begin{trivlist}
\item[\hskip \labelsep {\bfseries #1}\hskip \labelsep {\bfseries #2.}]}{\end{trivlist}}
\newenvironment{corollary}[2][Corollary]{\begin{trivlist}
\item[\hskip \labelsep {\bfseries #1}\hskip \labelsep {\bfseries #2.}]}{\end{trivlist}}
\theoremstyle{definition}
\newtheorem{definition}{Definition}[section]
\theoremstyle{definition}
\newtheorem{lemma}{Lemma}[section]
\theoremstyle{definition}
\newtheorem{exmp}{Example}[section]
\theoremstyle{definition}
\newtheorem{remark}{Remark}[section]
%-------------------------------------- 
\begin{document}

\title{Math Problem Set 2}
\author{Matthew Brown\\ 
OSM Boot Camp 2018} %if necessary, replace with your course title
 
\maketitle
 
\begin{problem}{3.1} There are two parts: 
\begin{itemize}
\item (i)
\begin{align*}
||x+y||^2 -||x-y||^2 &= \langle x+y,x+y \rangle - \langle x-y,x-y \rangle \\
&= \langle x+y,x \rangle + \langle x+y,y \rangle - (\langle x-y, x \rangle + \langle x-y, -y\rangle) \\
&= \langle x,x+y \rangle + \langle y,x+y \rangle - (\langle x, x-y \rangle + \langle -y,x-y\rangle \\
&= \langle x, x \rangle + \langle x, y\rangle + \langle y,x \rangle + \langle y,y \rangle - \langle x, x \rangle - \langle x, -y \rangle - \langle -y,x \rangle - \langle -y,-y \rangle \\
&= 4\langle x,y \rangle \text{ after some easy manipulations} 
\end{align*}
So it is clear that $\frac{1}{4}(||x+y||^2 -||x-y||^2) = \langle x, y \rangle $
\item (ii)
As above, 
\begin{align*}
||x+y||^2 + ||x-y||^2 &= \langle x+y,x+y \rangle + \langle x-y,x-y \rangle \\
&= \langle x+y,x \rangle + \langle x+y,y \rangle + (\langle x-y, x \rangle + \langle x-y, -y\rangle) \\
&= \langle x,x+y \rangle + \langle y,x+y \rangle + (\langle x, x-y \rangle + \langle -y,x-y\rangle \\
&= \langle x, x \rangle + \langle x, y\rangle + \langle y,x \rangle + \langle y,y \rangle + \langle x, x \rangle + \langle x, -y \rangle + \langle -y,x \rangle + \langle -y,-y \rangle \\
&= 2 \langle x, x \rangle + 2 \langle y, y, \rangle \text{ after some easy manipulations} \\
&= 2(||x||^2 + ||y||^2) 
\end{align*}
So it is clear that $\frac{1}{2}(||x+y||^2 + ||x-y||^2) = ||x||^2 + ||y||^2$
\end{itemize}
\end{problem}

\begin{problem}{3.2}
Latex code from kendra:
$\frac{1}{4} (||x+y||^2 - ||x-y||^2 + i ||x-iy||^2 - i ||x+iy||^2)$ \\ 
= $\frac{1}{4} (||x+y||^2 - ||x-y||^2)  + \frac{1}{4} (i ||x-iy||^2 - i ||x+iy||^2)$ \\
= $<x,y> + \frac{1}{4} (i\sqrt{<x-iy, x-iy>}^2 - i \sqrt{<x+iy, x+iy>}^2)$ by 3.1.i\\
= $<x,y> + \frac{1}{4} (i<x-iy, x-iy> - i<x+iy, x+iy>)$ \\
= $<x,y> + \frac{1}{4} (i(<x, x-iy> + <-iy, x-iy>) - i(<x, x+iy> +<iy, x+iy>))$ \\
= $<x,y> + \frac{1}{4} (i(<x, x> + <x, -iy> + <-iy, x> + <-iy,-iy>) \\ \indent - i(<x, x> +< x, iy> +<iy, x> +<iy, iy>))$ \\
= $<x,y> + \frac{1}{4} (i(<x, x> + i<x, y> + i<y, x> - <y,y>) \\ \indent - i(<x, x> -i < x, y> -i <y, x> -<y, y>))$ \\
= $<x,y> + \frac{1}{4} (i<x, x> -<x, y> -<y, x> - i<y,y> \\ \indent - i<x, x> + < x, y> + <y, x> +i<y, y>)$ \\
= $<x,y> + \frac{1}{4} ( -<x,y>  + < x, y>)  $ \\
= $<x,y>$ \\
\end{problem}

\begin{problem}{3.3}
Let $\theta$ be the angle in question. Recall that $$ cos(\theta) - \frac{\langle f, g \rangle}{||f|| \cdot ||g||} $$
\begin{itemize}
\item (i)
\begin{align*}
&\langle f,g \rangle = \int_0^1 fg dx = \int_0^1 x^6 dx = \bigg(\frac{x^7}{7} \bigg) \bigg\vert_0^1 = \frac{1}{7} \\
&||f||^2 = \int_0^1 f^2 dx = \int_0^1 x^2 dx = \bigg(\frac{x^3}{3} \bigg) \bigg\vert_0^1 = \frac{1}{3} \\
&||g||^2 = \int_0^1 g^2 dx = \int_0^1 x^10 dx = \bigg(\frac{x^{11}}{11}\bigg) \bigg\vert_0^1 = \frac{1}{11}
\end{align*}
Thus, we see that 
\begin{equation}
cos(\theta) = \frac{\frac{1}{7}}{(\frac{1}{3} \cdot \frac{1}{11}) ^\frac{1}{2}} = \frac{33^{\frac{1}{2}}}{7}
\end{equation}
And (1) implies that $\theta \approx 35^{\circ}$. 
\item (ii)
\begin{align*}
&\langle f,g \rangle = \int_0^1 fg dx = \int_0^1 x^6 dx = \bigg(\frac{x^7}{7} \bigg) \bigg\vert_0^1 = \frac{1}{7} \\
&||f||^2 = \int_0^1 f^2 dx = \int_0^1 x^4 dx = \bigg(\frac{x^5}{8} \bigg) \bigg\vert_0^1 = \frac{1}{5} \\
&||g||^2 = \int_0^1 g^2 dx = \int_0^1 x^8 dx = \bigg(\frac{x^{9}}{9}\bigg) \bigg\vert_0^1 = \frac{1}{9}
\end{align*}
Thus, we see that 
\begin{equation}
cos(\theta) = \frac{\frac{1}{7}}{(\frac{1}{5} \cdot \frac{1}{9}) ^\frac{1}{2}} = \frac{45^{\frac{1}{2}}}{7}
\end{equation}
And (1) implies that $\theta \approx 17^{\circ}$
\end{itemize}
\end{problem}

\begin{problem}{3.8} There are four parts
\begin{itemize}
\item (i)
\begin{proof}
This is just a matter of checking all the relevant details. \\
Norms = 1:
\begin{align*}
&||cos(t)||^2 = \frac{1}{\pi} \int_{-\pi}^{\pi} cos^2(t) dt = \frac{1}{\pi}\frac{cos(x)sin(x) + x}{2} \big\vert^{\pi}_{-\pi} = 1 \\
&||sin(t)||^2 = \frac{1}{\pi} \int_{-\pi}^{\pi} sin^2(t) dt = \frac{1}{\pi}\frac{-sin(2x)+2x}{4} \big\vert^{\pi}_{-\pi} = 1 \\
&||cos(2t)||^2 = \frac{1}{\pi} \int_{-\pi}^{\pi} cos^2(2t) dt = \frac{1}{\pi}\frac{sin(4x)+4x}{8} \big\vert^{\pi}_{-\pi} = 1 \\
&||sin(2t)||^2 = \frac{1}{\pi} \int_{-\pi}^{\pi} sin^2(2t) dt = \frac{1}{\pi}\frac{-sin(4x)+4x}{8} \big\vert^{\pi}_{-\pi} = 1
\end{align*}
Inner Products = 1:
\begin{align*}
&\langle cos(t), sin(t) \rangle = \frac{1}{\pi} \int_{-\pi}^{\pi} cos(t)sin(t) dt = \frac{sin^2(x)}{2} \big\vert^{\pi}_{-\pi} = sin^2(\pi) - sin^2(-\pi) = 0 
\end{align*}
The proof is completed by checking all the other inner products similarly.
\end{proof}
\item (ii) 
\begin{align*}
&||t||^2 = \int_{-\pi}^{\pi} t^2 dt = \frac{t^3}{3} \big\vert^{\pi}_{-\pi} = \frac{\pi^3}{3} - \frac{(-\pi)^3}{3} = \frac{2\pi^3}{3} \\
&||t|| = \bigg( \frac{2\pi^3}{3} \bigg)^{\frac{1}{2}}
\end{align*}
\item(iii) 
\begin{align*}
proj_X(cos(3t)) &= \langle sin(t), cos(3t)\rangle sin(t) + \langle cos(t), cos(3t) \rangle cos(t) \\
&+ \langle sin(2t), cos(3t) \rangle sin(2t) + \langle cos(2t), cos(3t) \rangle cos(2t) \\
&= 0 + 0 + 0 + 0 = 0
\end{align*}
(We see that $cos(3t)$ is orthogonal to $X$)
\item (iv) 
\begin{align*}
proj_X(t) = 0 + 2sin(t) + 0 - sin(2t)
\end{align*}
\end{itemize}
\end{problem}

\begin{problem}{3.9} 
\begin{proof}
In $\mathbb{R}^2$, a rotation about the origin by arbitrary angle $\theta$ can be described by the matrix 
$$
M = 
\begin{bmatrix}
cos(\theta) & - sin(\theta) \\
sin(\theta) & cos(\theta)
\end{bmatrix}
$$
So that $M\bigg(\begin{pmatrix} x \\ y \end{pmatrix}\bigg) = (\begin{smallmatrix} xcos(\theta) - ysin(\theta) \\ xsin(\theta) + ycos(\theta) \end{smallmatrix}) $.
Let $a = (\begin{smallmatrix} a_1 \\ a_2 \end{smallmatrix}), b= (\begin{smallmatrix} b_1 \\ b_2 \end{smallmatrix}) \in \mathbb{R}^2$. Then 
\begin{align*}
\langle M(a), M(b) \rangle &= \bigg\langle \begin{pmatrix} a_1cos(\theta) - a_2sin(\theta) \\ a_1sin(\theta) + a_2cos(\theta) \end{pmatrix}, \begin{pmatrix} b_1cos(\theta) - b_2sin(\theta) \\ b_1sin(\theta) + b_2cos(\theta) \end{pmatrix} \bigg\rangle \\
&= a_1b_1cos^2(\theta)+a_2b_2sin^2(\theta) + a_1b_1sin^2(\theta) + a_2b_2cos^2(\theta \\
&= a_1b_1+a_2b_2 = \langle a, b \rangle
\end{align*}
So we see that $M$ is an orthonormal operator. (N.b that this is a terribly inefficient way to prove this - I should have just shown that the columns of $M$ were orthonormal!
\end{proof}
\end{problem}
\begin{problem}{3.10} Recall that taking the Hermitian is flipping rows and columns and taking the conjugate.
\begin{itemize}
\item (i) \begin{proof}
We need to show both directions. \\
$\Rightarrow$: Let $Q$ be an orthonormal matrix. Then $\langle m, n \rangle = \langle Qm, Qn \rangle \implies m^H n = (Qm)^H Qn = m^H Q^H Qn.$ And because $m$ and $n$ were arbitrarily chosen, the only way that this equality holds is if $Q^HQ = I$, and this gives us that $QQ^H = Q$ since left inverse $\implies$ right inverse (for square matrices).  \\
$\Leftarrow$: Let $Q$ be a matrix so that $Q^HQ = I$. Then consider $\langle Qm, Qn \rangle = (Qm)^H Qn = m^HQ^HQn = m^Hn = \langle m, n \rangle$.
\end{proof}
\item (ii)
\begin{proof} This is pretty easy:
$$
||x|| = \sqrt[2]{\langle x, x \rangle} = \sqrt[2]{\langle x, x \rangle} = ||Qx||
$$
\end{proof}
\item (iii) \begin{proof}
Assume Q is orthonormal. Then $QQ^H=Q^HQ = I \implies Q^H = Q^{-1}$. I'll prove the following short lemma.
\begin{lemma} For $Q$ orthonormal, $Q^H$ is orthonormal.
\end{lemma}
Recall that $(Q^H)^H = Q$, and see that 
$$
(Q^H)^HQ^H = Q^H(Q^H)^H = I 
$$ 
which proves the lemma. \\
And since $Q^{-1} = Q^H$, $Q^{-1}$ is orthonormal. 
\end{proof}
\item (iv) 
\begin{proof}
We'll examine the elements of the identity matrix element by element. First note that: 
$$
I_{ij} = \delta_{ij} = \begin{cases} 1 \text{ if } i = j \\ 
0 \text{ if } i \neq j 
\end{cases}
$$ Then we'll compare this to what we get when we multiply $QQ^H$, which we know is equal to $I$ in all its coordinates. First, though, for any matrix $A$, define $A^i$ to be the "ith row" of $A$ and $A_j$ to be the jth column. Then
$$ \delta_{ij} = (Q^HQ)_{ij} = (Q^H)^i Q_j = $$
Recall now that $(Q^H)^i = \bar{Q_i}$, by definition of the Hermitian. But now we see that 
$$\langle \bar{Q_i}, q_j \rangle = \delta_{ij} $$
and the columns of $Q$ are orthonormal.
\end{proof}
\item (v) 
\begin{proof}
Consider an orthonormal matrix $Q \in M_n(\mathbb{R})$. 
$$ \text{ det}(Q) \text{ det}(Q^H) = \text{ det}(QQ^H) = \text{ det}(I) = 1 
$$
and det$Q) =$ det$(Q^H) \implies$ det$(Q) = 1$ or $-1$. \\
A counterexample to the converse would be the matrix $M = (\begin{smallmatrix} 2 & 0 \\ 0 & \frac{1}{2} \end{smallmatrix}) $. det $B = 1$, but $Be_1 = 2e_1$ and $||e_1|| \neq ||2e_1|| = ||Be_1||$ which violates what we proved in (ii).
\end{proof}
\item(vi) \begin{proof}
This is also quite short:

$$
(Q_1Q_2)(Q_1Q_2)^H = Q_1Q_2Q_2^HQ_1^H = Q_1Q_1^H = I
$$
And we also get the left inverse by properties of inverses.
\end{proof}
\end{itemize}
\end{problem}
\begin{problem}{3.11}
Suppose that $x_1, ... , x_n$ is as et of linearly $\textit{dependent}$ vectors. Let's apply Gram-Schmidt. Eventually, we will arrive at a vector $x_k$ which is linearly dependent upon $x_1, ..., x_{k-1}$. But then, if $X = span(x_1, ..., x_{k-1}) $, then $x_k \in X$ and $p_{k-1} = proj_X(x_k) = x_k$, which forces $q_k = 0$. In the end, if we throw out all these zeroes, we still get an orthonormal basis $q_1, ... q_m$ of $X$ where $m = \text{dim} X$. 
\end{problem}


\begin{problem}{3.16} Solved this with Albi!
\begin{itemize}
\item (i) Let $A\in\mathbb M_{mxn}$ where $\text{rank}(A)=n\leq m$.
Then there exist orthonormal $Q\in\mathbb M_{mxm}$ and
upper triangular $R\in\mathbb M_{mxn}$ such that $A=QR$.
Since $\tilde{Q}=-Q$ is still orthonormal ($-Q(-Q)^H=-Q(-Q^H)=QQ^H=I$ 
and similarly one shows $(-Q)^H(-Q)=I$)
and $\tilde{R}=-R$ is still upper triangular, 
$A=QR=\tilde{Q}\tilde{R}$.
Therefore QR-decomposition is not unique.
\item (ii) Suppose now that $A$ is invertible and can be decomposed into 
two different QR decompositions: $QR$ and $\tilde{Q}\tilde{R}$,
where the diagonal entries of $R$ and $\tilde{R}$ are strictly positive.
Then both $R$ and $\tilde{R}$ are invertible and we conclude that
$\tilde{R}^{-1}R=Q^H\tilde{Q}$.
Since $R$ and $\tilde{R}$ are upper triangular, so is the LHS of the previous equation.
On the other hand, since $Q$ and $\tilde{Q}$ are orthonormal, so is the RHS.
Therefore $\tilde{R}^{-1}R=I$ and, by unicity of the inverse, we conclude that $R=\tilde{R}$,
and so $Q=\tilde{Q}$.
\end{itemize}
\end{problem}

\begin{problem}{3.17} \begin{proof}
\begin{align*}
A^HAx = Ab \\
\iff (\hat{Q} \hat{R})^H \hat{Q} \hat{R} x = (\hat{Q}\hat{R})^H b \\
\iff (\hat{Q} \hat{R})^H \hat{Q} \hat{R} x \\
\iff \hat{R}^H \hat{Q}^H \hat{Q} \hat{R} x = \hat{R}^H \hat{Q}^Hb
\end{align*}
Multiply both sides by $\hat{R}^{H^{-1}}$ and we see that it is equivalent to $\hat{R}x = \hat{Q}^Hb$
\end{proof}
\end{problem}

\begin{problem}{3.23}
\begin{proof}
$$||x|| - ||y|| = ||x||+ ||-y|| \leq ||x-y|| $$ and 
$$||y|| - ||x|| = ||y||+ ||-x|| \leq ||y-x|| = ||x-y||$$
together imply
$\big| ||x|| - ||y|| \big| \leq ||x-y|| $
\end{proof}
\end{problem}

\begin{problem}{3.24}This is procedural, so I used Albi's latex code
\end{problem}
(i)
Since $|f(t)|\geq 0$ for every $t$, so is $\int_a^b|f(t)|dt$.
In addition, if $f=0$, then $\int_a^b|f(t)|dt=0$.
On the other hand, if $\int_a^b|f(t)|dt=0$ and $|f(t)|\geq 0$, it must be that
$|f(t)|=0$ for all t, implying that $f=0$.
Now take a constant $c\in\mathbb F$, then 
$\int_a^b|cf(t)|dt=\int_a^b|c||f(t)|dt=|c|\int_a^b|f(t)|dt$,
since $c$ does not depend on $t$.
Finally, take $g\in C([a, b]; \mathbb F)$.
Since $|f(t)+g(t)|\leq|f(t)|+|g(t)|$ for all $t$ and the integral is a linear operator,
we have that $\int_a^b|f(t)+g(t)|dt\leq\int_a^b|f(t)|dt + \int_a^b|g(t)|dt$.

(ii)
Since $|f(t)|^2\geq 0$ for every $t$, so is $\int_a^b|f(t)|^2dt$ and its square root.
In addition, if $f=0$, then $|f(t)|^2=0$ for all $t$ and $\sqrt{\int_a^b|f(t)|^2dt}=0$.
On the other hand, if $\sqrt{\int_a^b|f(t)|^2dt}=0$, 
then $\int_a^b|f(t)|^2dt=0$ and since $|f(t)|^2\geq 0$ for all $t$, 
it must be that $|f(t)|^2=0$ for all t, implying that $f=0$.
Now take a constant $c\in\mathbb F$, then 
$\sqrt{\int_a^b|cf(t)|^2dt}=\sqrt{\int_a^b|c|^2|f(t)|^2dt}=|c|\sqrt{\int_a^b|f(t)|^2dt}$,
since $c$ does not depend on $t$.
Finally, take $g\in C([a, b]; \mathbb F)$.
Since $|f(t)+g(t)|\leq|f(t)|+|g(t)|$ for all $t$, 
$x\mapsto x^2$ and $x\mapsto\sqrt{x}$ are monotonically increasing for nonnegative $x$ 
and the integral is a linear operator,
we have that $\sqrt{\int_a^b|f(t)+g(t)|^2dt}\leq\sqrt{\int_a^b|f(t)|^2dt + \int_a^b|g(t)|^2dt}
\leq||f||_{L2}+||g||_{L2}$.

(iii)
Since $|f(x)|\geq 0$ for all $x$, so is the $\sup_{x\in[a, b]}|f(x)|$.
In addition, if $f=0$, then $\sup_{x\in[a, b]}|f(x)|$ is also zero.
On the other hand, since $|f(x)|\geq 0$ for all $x$, $0\leq\sup_{x\in[a, b]}|f(x)|=0$
implies that we must have $f=0$.
Now take a constant $c\in\mathbb F$, then
$\sup_{x\in[a, b]}|cf(x)|=\sup_{x\in[a, b]}|c||f(x)|=|c|\sup_{x\in[a, b]}|f(x)|$.
Finally, take $g\in C([a, b];\mathbb F)$.
Since $|f(x)+g(x)|\leq|f(x)|+|g(x)|$ for all $x$, we have that
$\sup_{x\in[a, b]}|f(x)+g(x)|\leq\sup_{x\in[a, b]}\{|f(x)|+|g(x)|\}
\leq\sup_{x\in[a, b]}|f(x)|+\sup_{x\in[a, b]}|g(x)|$.


\begin{problem}{3.26} 
First I must show that topological equivalence is an equivalence relation.
\begin{proof}
I must show three things: (i) $x \sim x$. (ii) $x \sim y  \implies y \sim x$, (iii) $x \sim y$ and $y \sim z \implies x \sim z$. \\
(i) $ || \cdot ||_1 \sim || \cdot ||_1 $ trivially. Let $M \geq m$, then $m||x||_1 \leq || x ||_1 \leq M|| x ||_1$ for all x. \\
(ii) Also trivial: Suppose $|| \cdot ||_1 \sim || \cdot ||_2$. Then $m||x||_1 \leq || x ||_2 \leq M|| x ||_1$ for all x, which implies that $M^{-1}||x||_2 \leq ||x||_1 \leq m^{-1}||x||_2$. \\
(iii) Suppose $|| \cdot ||_1 \sim || \cdot ||_2 $, and $|| \cdot ||_2 \sim || \cdot ||_3$. Then $m||x||_1 \leq || x ||_2 \leq M|| x ||_1$ and $n||x||_2 \leq || x ||_3 \leq N|| x ||_2$. But we get from this that $mn||x||_1 \leq ||x||_3 \leq MN||x||_1$.
\end{proof}
Now I'll show that the $1, 2$, and $ \infty$ norms are topologically equivalent. 
\begin{proof}
(i) $|| \cdot ||_1 \sim || \cdot ||_2$: \\
If we think about the inner product as the standard dot-product, then we have $$(||x||_1)^2 = \sum_{i=1}^n \sum_{j=1}^n |x_i||x_j| \geq \sum_{i=1}^n x_i^2 = \langle x, x, \rangle = (||x||_2)^2$$ 
(the inequality comes because we simply threw out some positive terms on the left side). This implies that $||x||_1 \geq ||x||_2$. Moreover, 
$$
\sum_{i=1}^n|x_i|\cdot1\leq
\left(\sum_{i=1}^n|x_i|^2\right)^{1/2}\left(\sum_{i=1}^n1^2\right)^{1/2}=
\sqrt{n}\left(\sum_{i=1}^n|x_i|^2\right)^{1/2}
$$
so $||x||_2\leq ||x||_1 \leq \sqrt{n}||x||_2$. \\
(ii) $|| \cdot ||_\infty \sim || \cdot ||_2$

$$
||x||_\infty = max_{1 \leq i \leq n}\{ x_i \} = \sqrt[2]{(max_{1 \leq i \leq n} \{ x_i \})^2} \leq \sqrt[2]{\sum_{i=1}^n x_i} = ||x||_2
$$
and 
$$
||x||_2^2 = \sum_{i=1}^n |x_i|^2 \leq n max_i \{ x_i \} = (\sqrt{n}||x|_\infty|)^2 \implies ||x||_2 = \sqrt{n}||x||_\infty
$$
so $||x||\infty \leq ||x||_2 \leq \sqrt{n}||x||_\infty$
\end{proof}
\end{problem}

\begin{problem}{3.28} (Albi's latex code) \\
(i)
Notice that (applying the results of the previous exercise)
\begin{align*}
    \sup_{x\neq 0}\frac{||Ax||_1}{||x||_1}\leq
    \sup_{x\neq 0}\frac{||Ax||_1}{||x||_1}\leq
    \sqrt{n}\sup_{x\neq 0}\frac{||Ax||_2}{||x||_2},
\end{align*}
and
\begin{align*}
    \sup_{x\neq 0}\frac{||Ax||_1}{||x||_1}\geq
    \sup_{x\neq 0}\frac{||Ax||_2}{||x||_1}\geq
    \frac{1}{\sqrt{n}}\sup_{x\neq 0}\frac{||Ax||_2}{||x||_2}
\end{align*}
imply that $\frac{1}{\sqrt{n}}||A||_2\leq||A||_1\leq||A||_2$.

(ii)
Notice that
\begin{equation*}
    \sup_{x\neq 0}\frac{||Ax||_2}{||x||_2}\leq
    \sup_{x\neq 0}\frac{\sqrt{n}||Ax||_\infty}{||x||_\infty},
\end{equation*}
and
\begin{equation*}
    \sup_{x\neq 0}\frac{||Ax||_2}{||x||_2}\geq
    \sup_{x\neq 0}\frac{||Ax||_\infty}{\sqrt{n}||x||_\infty}.
\end{equation*}
\end{problem}

\begin{problem}{3.29}
I will prove two statements. \\

\textbf{The norm of an orthonormal matrix is 1:}
\begin{proof}
Let $Q$ be an orthonormal matrix. Then 
$$ ||Qx|| = ||x|| \implies sup_{x\neq 0} \frac{||Qx||}{||x||} = ||Q|| = 1 $$
\end{proof}

\textbf{If $R_x : M_n(\mathbb{F}) \to  \mathbb{F} , R_x(A) = Ax$, then $||R_x||=||x||$}:
\begin{proof}
The first step is to show $||R_x|| < ||x||$. $$||R_x|| = sup_{A\neq 0} \frac{||R_x(A)||}{||A||} = sup_{A\neq 0} \frac{||Ax||}{||A||} = sup_{A\neq 0} \frac{||Ax||\cdot ||x||}{||A||\cdot ||x||}$$
By Remark 3.5.12, $||Ax|| \leq ||A|| \cdot ||x|| \forall x \in \mathbb{F}^n$, so $$||R_x|| =sup_{A \neq 0} \frac{||Ax||\cdot ||x||}{||A||\cdot ||x||} \leq sup_{A \neq 0}\frac{||Ax|| \cdot ||x||}{||Ax||} = ||x|| $$
Now I'll show equality. For the $\leq$ above to be strict, we must have $ ||Ax|| < ||A||\cdot||A|| $ for all operators $A$ (because we're taking the supremum). $||x|| > 0$, so I can rearrange for the condition: 
$$ \frac{||Ax||}{||x||} < ||A||, \text{ for all operators } A, \text{ vectors } x $$
In other words, no $x$ achieves the supremum which is encoded in the definition of $||A||$. I will use the previous result to show that this will never hold. \\
Let $q_1 = e_1$ (or some other vector with norm 1). I can use the gram-schmidt algorithm to construct an orthonormal basis $q_1, ... q_n$ for $\mathbb{F}^n$. Let $Q$ be the matrix with these basis vectors as its columns. Then $Q$ is an orthonormal matrix. Specifically, $||Q|| = 1$ and it achieves $\frac{||Qx||}{||x||} = ||Q|| = 1$ at all nonzero $x$. \\
This shows that the inequality can never be strict, so we have $||R_x|| = ||x||$
\end{proof} 
\end{problem}

\begin{problem}{3.30} \begin{proof}
To show something is a norm, I must show three properties:
\begin{itemize}
\item \textit{Positivity:} This follows immediately from the positivity of the underlying matrix norm: $ ||A||_S = ||SAS^{-1}|| \geq 0$, with equality iff $SAS^{-1} = 0$, and there are no elements that are conjugate to $0$ other than itself so $A = 0$ in this case.  
\item \textit{Scalar Preservation:} I use linearity of S and the corresponding property for the matrix norm. 
$$ ||kA||_S = ||SkAS^{-1}|| = ||kSAS^{-1}|| = k||SAS^{-1}|| = k||A||_S $$
\item \textit{Triangle Inequality:} This also follows from the linearity of S. 
$$||(A+B)||_S = ||S(A+B)S^{-1}|| = ||SAS^{-1}+ SBS^{-1}|| \leq || ||SAS^{-1}| + ||SBS^{-1}|| = ||A||_S + ||B||_S $$
\end{itemize}
Finally, to see that it is a matrix norm, I must show that it is submultiplicative. And indeed 
$$||AB||_S = ||SABS^{-1}|| = ||SAS^{-1}SBS^{-1} || \leq ||SAS^{-1}||\cdot||SBS^{-1}|| = ||A||_S \cdot ||B||_S $$
\end{proof}
\end{problem}

\begin{problem}{3.37}
The first thing is to define the standard basis, which is $\mathcal{B} = \{1, x, x^2\}$ Evaluate $L$ on the basis vectors: 
$$L(1) = 0, L(x) = 1, L(x^2) = 2$$
Now, for $p \in V$, $p$ can be written as a linear combination of these basis vectors. So 
$$ L(p) = L(a_11 + a_2x + a_3x^2) = a_1L(1) +a_2 L(x) + a_3 L(x^2) = \langle (L(1)\cdot 1, L(x), L(x^2)) , (a_1, a_2, a_3) \rangle$$
which is the idea behind the Riesz Representation theorem. So we see that in this case, $q = (0, 1, 2)$... which squares with  what we know about derivatives.
\end{problem}

\begin{problem}{3.38} Let $\mathcal{B}$ as above.
$$
D = \begin{bmatrix}
0 & 1 & 0\\
0 & 0 & 2\\
0 & 0 & 0\\
\end{bmatrix}
$$
By following the same method as exercise 3.7.9, I see that 
$$D^* = -D = \begin{bmatrix}
0 & -1 & 0\\
0 & 0 & -2\\
0 & 0 & 0\\
\end{bmatrix}  $$
\end{problem}

\begin{problem}{3.39}
There are 4 things to show.
\begin{itemize}
\item (i)
\begin{proof}
\begin{align*}
\langle (S+T) v, w \rangle = \langle Sv, w \rangle + \langle Tv, w \rangle = \langle v, S^*w \rangle + \langle v, T^*w \rangle = \langle v, (S^* + T^*)w \rangle \\
\langle \alpha T^*v, w \rangle = \alpha \langle Tv, w \rangle= \alpha \langle v, T^*w \rangle = \langle v, \overline{\alpha}T^*w \rangle
\end{align*}
\end{proof}
\item (ii)
\begin{proof}
$$ \langle S^*v, w \rangle = \overline{\langle w, S^*v \rangle} = \overline{\langle Sw, v \rangle} = \langle v, Sw \rangle $$
\end{proof}
\item (iii)
\begin{proof}
$$\langle STv, w \rangle = \langle Tv, S^*w \rangle = \langle v, T^*S^*w \rangle$$
\end{proof}
\item (iv)
\begin{proof}
Consider the composition $T^*(T^{-1})^*$. 
$$
\langle 
T^*(T^{-1})^*x, y \rangle = \langle (T^{-1})^*x, Ty \rangle= \langle x, (T^{-1})Ty \rangle = \langle x, y \rangle
$$
Since the above is true for all $x, y$, we must have $T^*(T^{-1})^* = I $ 
\end{proof}
\end{itemize}
\end{problem}

\begin{problem}{3.40}
\end{problem}
\begin{itemize}
\item (i) (Considering $A$ as the operator)
$$
\langle AB, C \rangle = \text{ tr }(AB)^HC = \text{ tr } B^HA^HC = \langle B, A^HC \rangle
$$
\item (ii) 
\begin{align*}
\langle A_2, A_3A_1 \rangle = \text{ tr}(A_2^HA_3A_1) = \text{ tr}(A_1A_2^HA_3) = \text{ tr}(A_2A_1^HA_3) = \langle A_2A_1^*, A_3 \rangle
\end{align*}
\item (iii) (Albi)
Given $B,C\in\mathbb M_n(\mathbb F)$, we have $<B,AC-CA>=<B,AC>-<B,CA>$. 
Applying (ii) to the second term we get $<B,CA>=<BA^*,C>$.
On the other hand, 
\begin{equation*}
    <B,AC>=\text{tr}(B^HAC)=\text{tr}((A^HB)^HC)=<A^HB,C>=<A^*B,C>.
\end{equation*}
Putting all together we obtain that $T_A^*=T_{A^*}$
\end{itemize}

\begin{problem}{3.44}
\begin{proof}
 By the fundamental subspaces theorem, Ker$(A^H) =$ Range$(A)$. So we can reformulate the second possibility to: there exists $y \in \text{ Range}(A)^\perp : \langle y, b \rangle \neq 0 $. \\
 Consider now $p = proj_{\text{Range}(A)}b$. If $p = b$, then $b \in$ Range$(A)$ and we have the first case. Otherwise, the procedure creates a residual vector $r$, $r = b - p$. $r \in \text{ Range}(A)^\perp$, and 
$$\langle r, b \rangle = \overline{\langle p + r, r \rangle} = \overline{\langle p, r \rangle} + \overline{\langle r, r \rangle} = \langle p, b \rangle + \langle r, r \rangle = \langle r, r \rangle \neq 0$$
which is the second case. 
\end{proof}
\end{problem}

\begin{problem}{3.45}
Props to Albi for this idea - I think this is really slick (especially the last part)!
\begin{proof}
Double inclusion. First show that $\text{Skew}_n(\mathbb{R}) \subset \text{ Sym}_n^\perp(\mathbb{R})$. \\
Let $A \in {Skew}_n(\mathbb{R})$ Then, recalling the definitions of the spaces and the properties of trace,
$$\forall B \in \text{ Sym}_n(\mathbb{R})\text{, } \langle A, B \rangle = \text{ tr}(A^HB) = \text{ tr}(-AB) = \text{ tr}(-AB^H) = -\overline{\langle A, B \rangle}
$$ 
But since we are over $\mathbb{R}, \langle A, B \rangle = -\overline{\langle A, B \rangle} \implies \langle A,B \rangle = 0$ for all $B \in \text{ Sym}_n(\mathbb{R})$, so we see that $A \in \text{ Sym}_n(\mathbb{R})^{\perp}$ \\
Now I'll show that $\text{ Sym}_n^\perp(\mathbb{R}) \subset \text{Skew}_n(\mathbb{R})$. \\
Let $B \in \text{ Sym}_n(\mathbb{R})^\perp$. Then for any $A \in \text{ Sym}_n(\mathbb{R})$, We will examine $\langle B+B^T, A \rangle$.
$$
\langle B+B^T,A \rangle = \langle B, A\rangle + \langle B^T, A\rangle = 0 + \langle B^T, A\rangle 
$$
and $$
\langle B^T, A \rangle = \text{ tr}(BA) = \text{ tr}(BA^T) = \text{ tr}(A^TB) = \text{tr}(B^TA) = \langle B, A \rangle = 0
$$
so $\langle B+B^T, A \rangle = 0$ for all $A \in \text{ Sym}_n(\mathbb{R})$. But $B+B^T \in \text{ Sym}_n(\mathbb{R})$, so $||B+B^T|| = 0 \implies B+B^T=0 \implies B^T = -B$
\end{proof}
\end{problem}

\begin{problem}{3.46} Four Parts
\begin{proof}
\begin{itemize}
\item (i) $Ax \in$ Range$(A)$ by definition, and $x \in$ Ker$(A^HA) \implies A^H(Ax) = 0 \implies Ax \in$ Ker$A^H$.
\item (ii) Clearly Ker$(A) \subset $ Ker$(A^HA)$ since $A^H(0) = 0$. It remains to show that 
$$A^HAx = 0 \implies Ax = 0$$
Suppose $A^HAx = 0$. Then if we apply the operator $x^T$ as a row vector, we see that 
$$
x^HA^HAx = (Ax)^HAx = \langle Ax, Ax \rangle = 0
$$
and $\langle Ax, Ax \rangle = 0 \implies Ax = 0$ which is what we wanted to show.
\item (iii) $A$ and $A^HA$ both map to the n-dimensional spaces and Ker$(A)$ = Ker$(A^HA)$ by the above, so by Rank-Nullity,
\begin{align*}
n - \text{ dim Ker}(A) &= \text{ dim Range}(A) \\
n - \text{ dim Ker}(A^HA) &= \text{ dim Range}(A^HA)
\end{align*}
The left sides of the two equations are equal so the right sides must also be equal!
\item (iv) $A$ has linearly independent columns $\implies $ \\
$A$ has rank $n$ $\implies$ \\
$A^HA$ has (full) rank $n$ $\implies$ \\
$A^HA$ is non singular.
\end{itemize}
\end{proof}
\end{problem}

\begin{problem}{3.47}
\begin{proof}
\begin{itemize}
\item (i) $$P^2 = (A(A^HA^{-1})A^H)^2 = A(A^HA^{-1})A^HA(A^HA^{-1})A^H = AA^HAA^H = A(A^HA^{-1})A$$
\item (ii) \begin{align*}P^H = (A(A^HA^{-1})A^H)^H &= ((A(A^H))((A^{-1})A^H))^H \\
&=((A^{-1})A^H))^H((A(A^H))^H \\
&=A(A^{-1})^HAA^H
\end{align*}
\item (iii) Er... I'm not sure this is even true. Suppose $m < n$, then $P$ is $m$ by $m$ and cannot possibly have rank $n$? What am I missing here?
\end{itemize}
\end{proof}
\end{problem}

\begin{problem}{3.48} SO MANY PARTS AAAH!
\begin{proof}
\begin{itemize}
\item (i) I'll do it all at once... let $k \in \mathbb{R}, A,B \in M_n(\mathbb{R})$
\begin{align*}
P(k(A+B)) &= \frac{(k(A+B))+(k(A+B))^T}{2} \\
&= \frac{k(A+B)+(k(A^T+B^T)}{2} \\
&= \frac{k(A+A^T+B+B^T)}{2} \\
&= k(P(A)+P(B))
\end{align*}
\item (ii) 
\begin{align*}
P^2(A) = \frac{P(A)+P(A)^T}{2} = \frac{\frac{A+A^T}{2}+\frac{A+A^T}{2}}{2} = \frac{A+A^T}{2} = P(A)
\end{align*}
\item (iii) First see that $P(A) = P(A^T)$ (this is trivial). Then
\begin{align*}
\langle P(A), B \rangle = \text{ tr}(P(A)^TB) =\text{ tr}(\frac{A+A^T}{2} \cdot B) = \frac{\text{tr}(A^TB+AB)}{2} = \text{tr}(AB) \\
= \frac{\text{tr}(AB+AB^T)}{2} =  \text{ tr}(A \cdot \frac{B+B^T}{2}) = \text{ tr}(AP(B)) = \langle A, P(B) \rangle 
\end{align*}
\item (iv) 
\begin{align*}
A \in \text{ Ker}(P) \iff P(A) = 0 \iff A+A^T = 0 \iff A = -A^T \iff A \in \text{ Skew}_n(\mathbb{R})
\end{align*}
\item (v)
\begin{align*}
A \in \text{ Range}(P) &\iff \exists B : A = P(B) \\
&\iff \exists B: B + B^T = 2A \\
&\iff 2A \in \text{ Sym}_n(\mathbb{R}) \\
&\iff A \in \text{ Sym}_n(\mathbb{R}) 
\end{align*}
\item (vi) Copied latex code from Albi as this is just long and mechanical.
\begin{align*}
    &||A - P(A)||_F^2 = <A - P(A), A - P(A)> =
    <A - \frac{A + A^T}{2}, A - \frac{A + A^T}{2}> =\\
    &<\frac{A - A^T}{2}, \frac{A - A^T}{2}> = 
    \text{Tr}\left(\left(\frac{A - A^T}{2}\right)^T\frac{A - A^T}{2}\right)=\\
    &\text{Tr}\left(\frac{A^T - A}{2}\frac{A - A^T}{2}\right) = 
    \text{Tr}\left(\frac{A^TA - A^2 - (A^T)^2 + AA^T}{4}\right) =\\ 
    &\text{Tr}\left(\frac{A^TA - A^2 - A^2 + A^TA}{4}\right) =
    \text{Tr}\left(\frac{A^TA - A^2}{2}\right) = 
    \frac{\text{Tr}(A^TA) - \text{Tr}(A^2)}{2}.
\end{align*}
\end{itemize}
\end{proof}
\end{problem}
\begin{problem}{3.50}
I want to estimate the least squares solution for $Ax = b$ where:
\begin{align*}
A = \begin{bmatrix}
x_1^2 & y_1^2 \\
. & . \\
. & . \\
. & . \\
x_n^2 & y_n^2
\end{bmatrix} \text{,  }
x = \begin{bmatrix}
r \\
s
\end{bmatrix} \text{,  }
b = \begin{bmatrix}
1 \\
. \\
. \\
. \\
1
\end{bmatrix}
\end{align*}
The normal equation to solve is:
\begin{align*}
\begin{bmatrix}
\sum_{i=1}^n x_i^2 & \sum_{i=1}^n x_iy_i \\
\sum_{i=1}^n x_iy_i & \sum_{i=1}^n y_i^2
\end{bmatrix}
\begin{bmatrix}
r \\
s
\end{bmatrix} = \begin{bmatrix}
\sum_{i=1}^n x_i \\
\sum_{i=1}^n y_i
\end{bmatrix}
\end{align*} 
\end{problem}


\end{document}
