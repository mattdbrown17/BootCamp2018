\documentclass[12pt]{article}
 
\usepackage[margin=1in]{geometry} 
\usepackage{amsmath,amsthm,amssymb}
 
\newcommand{\N}{\mathbb{N}}
\newcommand{\Z}{\mathbb{Z}}
 
\newenvironment{theorem}[2][Theorem]{\begin{trivlist}
\item[\hskip \labelsep {\bfseries #1}\hskip \labelsep {\bfseries #2.}]}{\end{trivlist}}
\newenvironment{exercise}[2][Exercise]{\begin{trivlist}
\item[\hskip \labelsep {\bfseries #1}\hskip \labelsep {\bfseries #2.}]}{\end{trivlist}}
\newenvironment{problem}[2][Problem]{\begin{trivlist}
\item[\hskip \labelsep {\bfseries #1}\hskip \labelsep {\bfseries #2.}]}{\end{trivlist}}
\newenvironment{question}[2][Question]{\begin{trivlist}
\item[\hskip \labelsep {\bfseries #1}\hskip \labelsep {\bfseries #2.}]}{\end{trivlist}}
\newenvironment{corollary}[2][Corollary]{\begin{trivlist}
\item[\hskip \labelsep {\bfseries #1}\hskip \labelsep {\bfseries #2.}]}{\end{trivlist}}
\theoremstyle{definition}
\newtheorem{definition}{Definition}[section]
\theoremstyle{definition}
\newtheorem{lemma}{Lemma}[section]
\theoremstyle{definition}
\newtheorem{exmp}{Example}[section]
\theoremstyle{definition}
\newtheorem{remark}{Remark}[section]
%-------------------------------------- 
\begin{document}

\title{Math Problem Set 3}
\author{Matthew Brown\\ 
OSM Boot Camp 2018} %if necessary, replace with your course title
 
\maketitle
 
\begin{problem}{4.2}
Recall from the last homework that 
$$
D = \begin{bmatrix}
0 & 1 & 0\\
0 & 0 & 2\\
0 & 0 & 0\\
\end{bmatrix}
$$
This is an upper triangular matrix, so its has eigenvalues along the diagonal, so the only eigenvalue is 0. The eigenspace for this eigenvalue is the set of constant functions, which has dimension 1, so the geometric multiplicity is 1. The algebraic multiplicity, on the other hand, is 3, since 0 appears 3 times along the diagonal.
\end{problem}

\begin{problem}{4.4}
\begin{proof} Two parts: \\
\begin{itemize}
\item If $A = A^H$, then the diagonal elements of $A$ must be real (since they don't change when we take their complex conjugate). We know from exercise 4.3 that 
\begin{align*}
p(\lambda ) = \lambda^2 - \text{ tr}A\lambda + \text{ det} (A)
\end{align*}
By the quadratic formula, we can look at the "discriminant" $(\text{ tr}A)^2 - 4\text{ det} (A)$ to see if the roots are real. Now 
\begin{align*}
(\text{ tr}A)^2 - 4\text{ det} (A) = (A_11 + A_22)^2
\end{align*}
(NOT DONE BUT BRAIN IS BROKEN)
\end{itemize}
\end{proof}
\end{problem}

\begin{problem}{4.6}
\begin{proof}
Consider an upper triangular matrix $A$, with the elements $\lambda$ on the diagonal like so:
$$
A = \begin{bmatrix}
\lambda_1 & & \\
& \lambda_2 & \\
& & ... & & \\
& & & \lambda_k
\end{bmatrix}
$$
(Note of course that the $\lambda$s don't have to be unique.) A number c is an eigenvalue, iff the space Ker($cI - A)$ has some nonzero elements. We see that 
$$
cI - A = \begin{bmatrix}
c - \lambda_1 & & \\
& c - \lambda_2 & \\
& & ... & & \\
& & & c - \lambda_k
\end{bmatrix}
$$
For $c \neq \lambda_i$, the matrix is invertible and Ker$(cI - A) = \{0\} \implies c$ is not an eigenvalue. \\
For $c = \lambda_i$, some column of $cI -A$ will have zero on the diagonal. Choose the first such column, call it column $(cI-A)_r$. I claim that column $(cI-A)_r$ must be in the span of all the previous columns $(cI-A)_1,...,(cI-A)_{r-1}$.  \\
To see this, think of all the columns as $r-1$-dimensional vectors. I can do this because the coordinates in position $k>r$ are all 0, and thus they "don't matter" (technical term!). Since I'm thinking now of vectors in $\mathbb{F}^{r-1}$ and I notice that $(cI-A)_1, ... (cI-A)_{r-1}$ are linearly independent, I can say that these vectors form a basis for  $\mathbb{F}^{r-1}$. Therefore $(cI-A)_{r-n} \in \mathbb{F}^{r-1}$ is in the span of $ \{ (cI-A)_1, ... (cI-A)_{r-1} \}$... but there is no difficulty in passing to the whole column of $(cI-A)$! \\
That was long and there's probably a shorter way to do it (indeed I'm pretty sure the above isn't rigorous), but the point is that I've shown that the columns of $(cI-A)$ are not linearly independent. Thus $(cI-A)$ does not have full rank, so it's not injective, so its kernel must have nonzero elements, so $c$ is an eigenvalue.
\end{proof}
\end{problem}

\begin{problem}{4.8} Three Parts:
\begin{itemize}
\item
\begin{proof}
$S$ spans $V$ by definition. It remains to show that I can't write any of the elements as a linear combination of any of the others. I feel like should \textit{probably} be invoking that fourier stuff that Jan taught us right about now... but I'll wriggle my way out of that by doing a trick. Let's define for each function a 4-dimensional vector which evaluates the function at the points: $0 ,\dfrac{\pi}{4}, \dfrac{\pi}{3}$ and $\dfrac{\pi}{2}$. The vectors are: 
\begin{align*}
v_{sin(x)} = \begin{bmatrix}
sin(0) = 0 \\
sin(\dfrac{\pi}{4}) = .707 \\
sin(\dfrac{\pi}{3}) = .866 \\
sin(\dfrac{\pi}{2}) = 1
\end{bmatrix}
v_{cos(x)} = \begin{bmatrix}
1 \\
.707 \\
.5 \\
0
\end{bmatrix}
v_{sin(2x)} = \begin{bmatrix}
0 \\
1 \\
.866 \\
0
\end{bmatrix}
v_{cos(2x)} = \begin{bmatrix}
1 \\
0\\
-.5 \\
-1 
\end{bmatrix}
\end{align*}
I want to check to see if these vectors are linearly independent. One check is to see if a matrix with these vectors as columns row-reduces to the identity matrix, which it does. And if I can't express any of the functions as a linear combination of the others at these 4 points, than I sure as heck know that I can't express any of them as a linear combination of the others generally, so they are linearly independent.
\end{proof}
\item (ii) $$
D = \begin{bmatrix}
0 & -1 & 0 & 0 \\
1 & 0 & 0 & 0 \\
0 & 0 & 0 & -2 \\
0 & 0 & 2 & 0 
\end{bmatrix}
$$
\item (iii) The spaces $E_1 = \text{ span} \{ sin(x), cos(x) \}$ and $E_2 = \text{ span} \{ sin(2x), cos(2x) \}$ are complementary and $D$- invariant.
\end{itemize}
\end{problem}

\begin{problem}{4.13}
First step is to find the basis of eigenvectors... Eigenvalues are $1, .4$ with eigenvectors $(2, 1)$ and $(1, -1)$ So I want to change to this basis, and I'll use 
$$
P = \begin{bmatrix}
2 & 1 \\
1 & -1\\
\end{bmatrix}
$$
as my transition matrix. * (Check how to do change of basis lol) *
\end{problem}

\begin{problem}{4.15}
Let $\mathcal{B} = \{b_1, ..., b_n\}$ be an eigenbasis for $A$, with each $b_i$ corresponding to an eigenvalue $\lambda_i$. Consider the action of $f(A)$ on the vector $f(b_i)$. First I show that $Af(b_i) = \lambda_i f(b_i)$ \\
Proof of the above: 
\begin{align*}
Af(b_i) &= A(a_0 + a_1b_i + a_2b_i^2 +... +a_nb_i^n = a_0A + a_1Ab_i + a_2Ab_i^2 + ... +a_nAb_i^n \\
&= a_oA + a_1 \lambda_i b_i + a_2 \lambda_i^2 b_i +  ... + a_n \lambda_i^n b_i = f(\lambda_i)b_i
\end{align*} 
Then, see that
\begin{align*}
f(A)f(b_i) = (a_0 + a_1A + a_2A^2 + ... + a_nA^n)f(b_i) = a_0f(b_i) + a_1 \lambda_i f(b_i) + a_2 \lambda_i^2 f(b_i) + ... + \lambda_i^n f(b_i) = f(\lambda_i)f(b_i)
\end{align*}
\end{problem}

\begin{problem}{4.16}
\end{problem}

\begin{problem}{4.18}
\end{problem}

\begin{problem}{4.20}
\end{problem}

\begin{problem}{4.24}
\end{problem}

\begin{problem}{4.25}
\end{problem}

\begin{problem}{4.27}
\end{problem}

\begin{problem}{4.28}
\end{problem}

\begin{problem}{4.31}
\end{problem}

\begin{problem}{4.32}
\end{problem}

\begin{problem}{4.33}
\end{problem}

\begin{problem}{4.36}
\end{problem}

\begin{problem}{4.38}
\end{problem}


\end{document}
