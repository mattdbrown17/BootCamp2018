\documentclass[letterpaper,12pt]{article}
\usepackage{array}
\usepackage{threeparttable}
\usepackage{geometry}
\geometry{letterpaper,tmargin=1in,bmargin=1in,lmargin=1.25in,rmargin=1.25in}
\usepackage{fancyhdr,lastpage}
\pagestyle{fancy}
\lhead{}
\chead{}
\rhead{}
\lfoot{}
\cfoot{}
\rfoot{\footnotesize\textsl{Page \thepage\ of \pageref{LastPage}}}
\renewcommand\headrulewidth{0pt}
\renewcommand\footrulewidth{0pt}
\usepackage[format=hang,font=normalsize,labelfont=bf]{caption}
\usepackage{listings}
\lstset{frame=single,
  language=Python,
  showstringspaces=false,
  columns=flexible,
  basicstyle={\small\ttfamily},
  numbers=none,
  breaklines=true,
  breakatwhitespace=true
  tabsize=3
}
\usepackage{amsmath}
\usepackage{amssymb}
\usepackage{amsthm}
\usepackage{harvard}
\usepackage{setspace}
\usepackage{float,color}
\usepackage[pdftex]{graphicx}
\usepackage{hyperref}
\usepackage{mathrsfs}
\let\vec\mathbf
\hypersetup{colorlinks,linkcolor=red,urlcolor=blue}
\theoremstyle{definition}
\newtheorem{theorem}{Theorem}
\newtheorem{acknowledgement}[theorem]{Acknowledgement}
\newtheorem{algorithm}[theorem]{Algorithm}
\newtheorem{axiom}[theorem]{Axiom}
\newtheorem{case}[theorem]{Case}
\newtheorem{claim}[theorem]{Claim}
\newtheorem{conclusion}[theorem]{Conclusion}
\newtheorem{condition}[theorem]{Condition}
\newtheorem{conjecture}[theorem]{Conjecture}
\newtheorem{corollary}[theorem]{Corollary}
\newtheorem{criterion}[theorem]{Criterion}
\newtheorem{definition}[theorem]{Definition}
\newtheorem{derivation}{Derivation} % Number derivations on their own
\newtheorem{example}[theorem]{Example}
\newtheorem{exercise}[theorem]{Exercise}
\newtheorem{lemma}[theorem]{Lemma}
\newtheorem{notation}[theorem]{Notation}
\newtheorem{problem}[theorem]{Problem}
\newtheorem{proposition}{Proposition} % Number propositions on their own
\newtheorem{remark}[theorem]{Remark}
\newtheorem{solution}[theorem]{Solution}
\newtheorem{summary}[theorem]{Summary}
\bibliographystyle{aer}
\newcommand\ve{\varepsilon}
\newcommand\boldline{\arrayrulewidth{1pt}\hline}
\newcolumntype{C}{>{$}c<{$}}

%-------------------------------------- 
\begin{document}

\title{Math Problem Set 5}
\author{Matthew Brown\\ 
OSM Boot Camp 2018} %if necessary, replace with your course title
 
\maketitle
\textbf{MASSIVE CREDIT} to Reiko Laski for creating the templates for the dictionaries. Indeed some of these simplex writeups are. And apologies for the weird numbering formatting.
\begin{problem}{8.1}
Jupyter notebook
\end{problem}

\begin{problem}{8.2}
Jupyter Notebook
\end{problem}

\begin{problem}{8.3}
Let $x_1$ denote the number of Barb soldiers sold and $x_2$ the number of Joey dolls. The optimization problem then is:
\begin{align*}
\text{max}_{x_1, x_2} & (4 x_1 + 3 x_2) \\
\text{subject to }
& 2x_1 + 2 x_2 \leq 300 \\
& 15x_1 + 10x_2 \leq 1800 \\
& x_2 \leq 200 \\
x_1, x_2 \geq 0
\end{align*}
\end{problem}

\begin{problem}{8.4}
Let $x_{ij}$ be the units that flow between node i and j. The optimization problem is:
\begin{align*}
\text{min}_{x_{ij}, i \neq j} & 2x_{AB} + 5x_{AD} +4 x_{DE} + 3 x_{EF} + 2 x_{CF} + 5x_{BC} + 2 x_{BD} + 7x_{BE} + 9 x_{BF} \\
\text{subject to }& x_{AB} + x_{AD} = 10\\
& x_{BD} + x_{BE} +x_{BF} + x_{BC} - x_{AB} = 1 \\
& x_{CF} - x_{BC} = -2 \\
& - (x_{EF} + x_{BF} + x_{CF}) = -10 \\
& x_{EF} - (x_{DE} + x_{BE}) = 4 \\
& x_{DE} - (x_{AD} + x_{BD}) = -3 \\
& 0 \leq x_{ij} \leq 6
\end{align*}
\end{problem}

\begin{problem}{8.5} \\
(i)
\begin{align*}
  &\text{maximize} \ \ 3x_1 + x_2 \\
  &\text{subject to} \ \ x_1 + 3x_2 + w_1 = 15 \\
  &\qquad \qquad \ \ \  2x_1 + 3x_2 + w_2 = 18 \\
  &\qquad \qquad \ \ \  x_1 - x_2 + w_3 = 4 \\
  &\qquad \qquad \ \ \  x_1, x_2, w_1, w_2, w_3 \geq 0
\end{align*}
\begin{center}
  \def\arraystretch{1.2}
  \begin{tabular}{ C C C C C C C }
    \zeta & = & & & 3x_1 & + & x_2 \\
    \hline
    w_1 & = & 15 & - & x_1 & - & 3x_2 \\
    w_2 & = & 18 & - & 2x_1 & - & 3x_2 \\
    w_3 & = & 4 & - & x_1 & + & x_2 \\
    \hline \hline
    \zeta & = & 12 & + & 4x_2 & - & 3w_3 \\
    \hline
    w_1 & = & 11 & - & 4x_2 & + & w_3 \\
    w_2 & = & 10 & - & 5x_2 & + & 2w_3 \\
    x_1 & = & 4 & + & x_2 & - & w_3 \\
    \hline \hline
    \zeta & = & 20 & - & \tfrac{4}{5}w_2 & - & \tfrac{7}{5}w_3 \\
    \hline
    w_1 & = & 3 & + & \tfrac{4}{5}w_2 & - & \tfrac{3}{5}w_3 \\
    x_2 & = & 2 & - & \tfrac{1}{5}w_2 & + & \tfrac{2}{5}w_3 \\
    x_1 & = & 6 & - & \tfrac{1}{5}w_2 & - & \tfrac{3}{5}w_3\\
    \hline
  \end{tabular}
\end{center}
Optimizer: $(6, 2)$ \\
Optimum value: $20$

(ii)
\begin{align*}
  &\text{maximize} \ \ 4x + 6y \\
  &\text{subject to} \ \ -x + 3x_2 + w_1 = 11 \\
  &\qquad \qquad \ \ \  x + y + w_2 = 27 \\
  &\qquad \qquad \ \ \  2x + 5y + w_3 = 90 \\
  &\qquad \qquad \ \ \  x, y, w_1, w_2, w_3 \geq 0
\end{align*}
\begin{center}
  \def\arraystretch{1.2}
  \begin{tabular}{ C C C C C C C }
    \zeta & = & & & 4x & + & 6y \\
    \hline
    w_1 & = & 11 & + & x & - & y \\
    w_2 & = & 27 & - & x & - & y \\
    w_3 & = & 90 & - & 2x & - & 5y \\
    \hline \hline
    \zeta & = & 66 & + & 10x & - & 6w_1 \\
    \hline
    y & = & 11 & + & x & - & w_1 \\
    w_2 & = & 16 & - & 2x & + & w_1 \\
    w_3 & = & 35 & - & 7x & + & 5w_1 \\
    \hline \hline
    \zeta & = & 116 & + & \tfrac{8}{7}w_1 & - & \tfrac{10}{7}w_3 \\
    \hline
    y & = & 16 & - & \tfrac{2}{7}w_1 & - & \tfrac{1}{7}w_3 \\
    w_2 & = & 6 & - & \tfrac{3}{7}w_1 & + & \tfrac{2}{7}w_3 \\
    x & = & 5 & + & \tfrac{5}{7}w_1 & - & \tfrac{1}{7}w_3 \\
    \hline \hline
    \zeta & = & 132 & - &\tfrac{8}{3}w_2 & - & \tfrac{2}{7}w_3 \\
    \hline
    y & = & 12 & + & \tfrac{2}{3}w_2 & - & \tfrac{1}{3}w_3 \\
    w_1 & = & 14 & - & \tfrac{7}{3}w_2 & + & \tfrac{2}{3}w_3 \\
    x & = & 15 & - & \tfrac{5}{3}w_2 & + & \tfrac{1}{3}w_3 \\
    \hline
  \end{tabular}
\end{center}
Optimizer: $(15, 12)$ \\
Optimum value: $132$ \\
\end{problem}

\begin{problem}{8.6}
\begin{align*}
  &\text{maximize} \ \ 4b + 3j \\
  &\text{subject to} \ \ 15b + 10j + w_1 = 1800 \\
  &\qquad \qquad \ \ \  2b + 2j + w_2 = 300 \\
  &\qquad \qquad \ \ \  j + w_3 = 200 \\
  &\qquad \qquad \ \ \  b, j, w_1, w_2, w_3 \geq 0
\end{align*}
\begin{center}
  \def\arraystretch{1.2}
  \begin{tabular}{ C C C C C C C }
    \zeta & = & & & 4b & + & 3j \\
    \hline
    w_1 & = & 1800 & - & 15b & - & 10j \\
    w_2 & = & 300 & - & 2b & - & 2j \\
    w_3 & = & 200 & - & j \\
    \hline \hline
    \zeta & = & 450 & + & b & - & \tfrac{3}{2}w_2 \\
    \hline
    w_1 & = & 300 & - & 5b & + & 5w_2 \\
    j & = & 150 & - & b & - & \tfrac{1}{2}w_2 \\
    w_3 & = & 50 & + & b & + & \tfrac{1}{2}w_2 \\
    \hline \hline
    \zeta & = & 510 & - & \tfrac{1}{5}w_1 & - & \tfrac{1}{2}w_2 \\
    \hline
    b & = & 60 & - & \tfrac{1}{5}w_1 & + & w_2 \\
    j & = & 90 & + & \tfrac{1}{5}w_1 & - & \tfrac{3}{2}w_2 \\
    w_3 & = & 110 & - & \tfrac{1}{5}w_1 & + & \tfrac{3}{2}w_2 \\
    \hline
  \end{tabular}
\end{center}
Optimal choice: $60$ GI Barb soldiers, $90$ Joey dolls \\
Maximal profit: $\$510$ \\
\end{problem}

\begin{problem}{8.7}

\begin{itemize}
\item (i) COME BACK TO
\item (ii)
\begin{align*}
  &\text{maximize} \ \ 5x_1 + 2x_2 \\
  &\text{subject to} \ \ 5x_1 + 3x_2 + x_3 = 15 \\
  &\qquad \qquad \ \ \  3x_1 + 5x_2 + x_4 = 15 \\
  &\qquad \qquad \ \ \  4x_1 - 3x_2 + x_5 = -12 \\
  &\qquad \qquad \ \ \  x_1, x_2, x_3, x_4, x_5 \geq 0
\end{align*}
Auxiliary problem:
\begin{align*}
  &\text{maximize} \ \ -x_0 \\
  &\text{subject to} \ \ 5x_1 + 3x_2 + x_3 - x_0 = 15 \\
  &\qquad \qquad \ \ \  3x_1 + 5x_2 + x_4 - x_0 = 15 \\
  &\qquad \qquad \ \ \  4x_1 - 3x_2 + x_5 - x_0 = -12 \\
  &\qquad \qquad \ \ \  x_0, x_1, x_2, x_3, x_4, x_5 \geq 0
\end{align*}
\begin{center}
  \def\arraystretch{1.2}
  \begin{tabular}{ C C C C C C C C C C C }
    \zeta & = & & & & & & - & x_0 \\
    \hline
    x_3 & = & 15 & - & 5x_1 & - & 3x_2 & + & x_0 \\
    x_4 & = & 15 & - & 3x_1 & - & 5x_2 & + & x_0 \\
    x_5 & = & -12 & - & 4x_1 & + & 3x_2 & + & x_0 \\
    \hline \hline
    \zeta & = &  -12 & - & 4x_1 & + & 3x_2 & - & x_5 \\
    \hline
    x_3 & = & 27 & - & x_1 & - & 6x_2 & + & x_5 \\
    x_4 & = & 27 & + & x_1 & - & 8x_2 & + & x_5 \\
    x_0 & = & 12 & + & 4x_1 & - & 3x_2 & + & x_5 \\
    \hline \hline
    \zeta & = & -\tfrac{15}{8} & - & \tfrac{29}{8}x_1 & - & \tfrac{3}{8}x_4 & - & \tfrac{5}{8}x_5 \\
    \hline
    x_3 & = & \tfrac{27}{4} & - & \tfrac{7}{4}x_1 & + & \tfrac{3}{4}x_4 & + & \tfrac{1}{4}x_5 \\
    x_2 & = & \tfrac{27}{8} & + & \tfrac{1}{8}x_1 & - & \tfrac{1}{8}x_4 & + & \tfrac{1}{8}x_5 \\
    x_0 & = & \tfrac{15}{8} & + & \tfrac{29}{8}x_1 & + & \tfrac{3}{8}x_4 & + & \tfrac{5}{8}x_5 \\
    \hline
  \end{tabular}
\end{center}
We see that the optimum for the auxillary problem is nonzero, so there is no way to make $x_0$ become $0$, and therefore we say that the original problem has no feasible points. \\

\item (iii) For this one there's actually no need to set up an auxillary problem - which is \textit{really} nice.

\begin{center}
  \def\arraystretch{1.2}
  \begin{tabular}{ C C C C C }
	\hline \hline    
    \zeta & = & & -3x_1 & + x_2 \\
    \hline
    w_1 & = & 4 & & -x_2 \\ 
    w_2 & = & 6 & + 2x_1 & -3x_2 \\
    \hline \hline
    \zeta & = & 2 & -\frac{7}{3}x_1 & -\frac{1}{3}w_2 \\
    \hline
    w_1 & = & 2 & -2x_1 & +w_2 \\
    x_2 & = & 2 & +\frac{2}{3}x_1 & -\frac{1}{3}w_2 \\
  \end{tabular}
\end{center}
The optimum value is 2, which is attained at (0, 2).
\end{itemize}
\end{problem}

\begin{problem}{8.8}
I'll go very simple for these example questions...
\begin{align*}
  &\text{maximize} \ \ -x - y - z \\
  &\text{subject to} \ \ -x + y \leq 1 \\
  &\qquad \qquad \ \ \ x - y \leq 1 \\
  &\qquad \qquad \ \ \ x - z \leq 1 \\
  &\qquad \qquad \ \ \ -x + z \leq 1 \\
  &\qquad \qquad \ \ \ x , y, z \geq 0
\end{align*}
gives an unbounded closed feasible region (it's like this rectangular prism corridor which shoots out from the origin), but the optimum is attained at (0, 0).
\end{problem}

\begin{problem}{8.9}
I'll re-use the same example function, but change the objective function.
\begin{align*}
&\text{maximize} \ \ x + y + z \\
  &\text{subject to} \ \ -x + y \leq 1 \\
  &\qquad \qquad \ \ \ x - y \leq 1 \\
  &\qquad \qquad \ \ \ x - z \leq 1 \\
  &\qquad \qquad \ \ \ -x + z \leq 1 \\
  &\qquad \qquad \ \ \ x, y ,z \geq 0
\end{align*}
I have the same feasible region, but now the minimum of the objective function is $(0, 0)$, and there is no maximum. (To see this note that the triple $(N, N, N)$ satisfies all constraints for any $N \in \mathbb{R}$)
\end{problem}

\begin{problem}{8.10}
Boy, this example I've picked really is a workhorse:
\begin{align*}
&\text{maximize} \ \ x + y + z \\
  &\text{subject to} \ \ -x + y \leq -1 \\
  &\qquad \qquad \ \ \ x - y \leq -1 \\
  &\qquad \qquad \ \ \ x - z \leq -1 \\
  &\qquad \qquad \ \ \ -x + z \leq -1 \\
  &\qquad \qquad \ \ \ x, y ,z \geq 0
\end{align*}
I've flipped all the constraints, and now I get an empty feasible region. This can be seen just by looking at the first two constraints - if $-x+y \leq -1$, then $x-y \geq 1 > -1 $, so we see that there are no points which satisfy both of the first two constraints.
\end{problem}

\begin{problem}{8.11}
I'll modify the workhorse by bounding the prism within two hyperplanes (two more constraints - the geometric intuiton is more clear if you think of $- x - y - z \leq -1$ as $- x - y - z \geq -1$)
\begin{align*}
&\text{maximize} \ \ x + y + z \\
  &\text{subject to} \ \ -x + y \leq 1 \\
  &\qquad \qquad \ \ \ x - y \leq 1 \\
  &\qquad \qquad \ \ \ x - z \leq 1 \\
  &\qquad \qquad \ \ \ -x + z \leq 1 \\
  &\qquad \qquad \ \ \ - x - y - z \leq -1 \\
  &\qquad \qquad \ \ \ x + y + z \leq 5 \\
  &\qquad \qquad \ \ \ x, y ,z \geq 0
\end{align*}
The auxillary problem is :
\begin{align*}
&\text{maximize} \ \ -x_0 \\
  &\text{subject to} \ \ -x + y + x_0 \leq 1 \\
  &\qquad \qquad \ \ \ x - y + x_0\leq 1 \\
  &\qquad \qquad \ \ \ x - z + x_0 \leq 1 \\
  &\qquad \qquad \ \ \ -x + z + x_0\leq 1 \\
  &\qquad \qquad \ \ \ - x - y - z + x_0 \leq -1 \\
  &\qquad \qquad \ \ \ x + y + z + x_0 \leq 5 \\
  &\qquad \qquad \ \ \ x, y ,z \geq 0
\end{align*}
If this problem solves with an optimum $-x_0 = 0$ (which it should) then the point we get will be a feasible point for the original problem.
\end{problem}

\begin{problem}{8.12}
\begin{align*}
  &\text{maximize} \ \ 10x_1 - 57x_2 - 9x_3 -24x_4 \\
  &\text{subject to} \ \ 0.5x_1 - 1.5x_2 - 0.5x_3 + x_4 + x_5 = 0 \\
  &\qquad \qquad \ \ \  0.5x_1 - 5.5x_2 - 2.5x_3 + 9x_4 + x_6 = 0 \\
  &\qquad \qquad \ \ \  x_1 + x_7 = 0 \\
  &\qquad \qquad \ \ \  x_1, x_2, x_3, x_4, x_5, x_6, x_7 \geq 0
\end{align*}
\begin{center}
  \def\arraystretch{1.2}
  \begin{tabular}{ C C C C C C C C C C C C C }
    \zeta & = & & & 10x_1 & - & 57x_2 & - & 9x_3 & - & 24x_4 \\
    \hline
    x_5 & = & & - & 0.5x_1 & + & 1.5x_2 & + &  0.5x_3 & - & x_4 \\
    x_6 & = & & - & 0.5x_1 & + & 5.5x_2 & + & 2.5x_3 & - & 9x_4 \\
    x_7 & = & 1 & - & x_1 \\
    \hline \hline
    \zeta & = & & - & 27x_2 & + & x_3 & - & 44x_4 & - & 20x_5 \\
    \hline
    x_1 & = & & & 3x_2 & + & x_3 & - & 2x_4 & - & 2x_5 \\
    x_6 & = & & & 4x_2 & + & 2x_3 & - & 8x_4 & + & x_5 \\
    x_7 & = & 1 & - & 3x_2 & - & x_3 & + & 2x_4 & + & 2x_5 \\
    \hline \hline
    \zeta & = & 1 & - & 30x_2 & - & 42x_4 & - & 18x_5 & - & x_7 & \\
    \hline
    x_1 & = & 1 & & & & & & & - & x_7 \\
    x_6 & = & 2 & - & 2x_2 & - & 4x_4 & + & 5x_5 & - & 2x_7\\
    x_3 & = & 1 & - & 3x_2 & + & 2x_4 & + & 2x_5 & - & x_7 \\
    \hline
\end{tabular}
\end{center}
The optimum is 1, which is attained at $(1, 0, 1, 0)$
\end{problem}

\begin{problem}{8.15}
\begin{proof}
Note that we have $x, y \succeq 0$. Thus we have the following implications:
\begin{align*}
A^Ty \succeq c \implies (A^Ty)^Tx \geq c^Tx \\
Ax \preceq b \implies (Ax)^Ty \leq b^Ty 
\end{align*}
And we see that
\begin{align*}
(Ax)^Ty = x^TA^Ty = (y^TA^Tx)^T = y^TA^Tx = (Ay)^Tx^*
\end{align*}
because all these are real numbers and thus equal to their transposes. We conclude by seeing
\begin{align*}
c^Tx \leq (A^Ty)^Tx = (A^Ty)^Tx \leq b^Tx
\end{align*}
\end{proof}
\end{problem}

\begin{problem}{8.17}
Let the primal problem be   
\begin{align*}
&\text{maximize}_x \ \ c^Tx \\
  &\text{subject to} \ \ Ax \preceq b \\
  &\qquad \qquad \ \ \ x \succeq 0
\end{align*}
then the dual is:
\begin{align*}
&\text{minimize}_y \ \ b^Ty \\
  &\text{subject to} \ \ A^Ty \succeq c \\
  &\qquad \qquad \ \ \ y \succeq 0
\end{align*}
Then by definition the dual of the dual is
\begin{align*}
&\text{maximize}_z \ \ c^Tz \\
  &\text{subject to} \ \ (A^T)^Tz \succeq b \\
  &\qquad \qquad \ \ \ z \succeq 0
\end{align*}
and since $(A^T)^T = A$, this is just the primal problem.
\end{problem}

\begin{problem}{8.18} \\
Solving the standard problem:
\begin{center}
  \def\arraystretch{1.2}
  \begin{tabular}{ C C C C C C C C C C C C C }
    \zeta & = & & & x_1 & + & x_2 \\
    \hline
    w_1 & = & 3 & - & 2x_1 & - & x_2 \\
    x_2 & = & 5 & - & x_1 & - & 3x_2 \\
    x_5 & = & 4 & - & 2x_1 & - & 3x_2 \\
    \hline \hline
    \zeta & = & \tfrac{3}{2} & + & \tfrac{1}{2}x_2 & - & \tfrac{1}{2}w_1 \\
    \hline
    x_1 & = & \tfrac{3}{2} & - & \tfrac{1}{2}x_2 & - & \tfrac{1}{2}w_1 \\
    w_2 & = & \tfrac{7}{2} & - & \tfrac{5}{2}x_2 & + & \tfrac{1}{2}w_1 \\
    w_3 & = & 1 & - & 2x_2 & + & w_1 \\
    \hline \hline
    \zeta & = & \tfrac{7}{4} & - & \tfrac{1}{4}w_1 & - & \tfrac{1}{4}w_1 \\
    \hline
    x_1 & = & \tfrac{5}{4} & - & \tfrac{3}{4}w_1 & + & \tfrac{1}{4}w_3 \\
    w_2 & = & \tfrac{9}{4} & - & \tfrac{3}{4}w_1 & + & \tfrac{5}{4}w_3 \\
    x_2 & = & \tfrac{1}{2} & + & \tfrac{1}{2}w_1 & - & \tfrac{1}{2}w_3 \\
    \hline
  \end{tabular}
\end{center}
The function attains its optimum $\tfrac{7}{4}$ at the point $(\tfrac{5}{4}, \tfrac{1}{2})$ \\
\begin{align*}
&\text{minimize} \ \ 3y_1 + 5y_2 + 4 y_3 \\
  &\text{subject to} \ \ 2y_1 + y_2 + 3y_3 \geq 1 \\
  &\qquad \qquad \ \ \ y_1 + 3y_2 + 3y_3 \geq 1\\
  &\qquad \qquad \ \ \ y_1, y_2, y_3 \geq 0
\end{align*}

Which in standard form is:

\begin{align*}
&\text{maximize} \ \ -3y_1 - 5y_2 - 4 y_3 \\
  &\text{subject to} \ \ - 2y_1 - y_2 - 3y_3 \leq -1 \\
  &\qquad \qquad \ \ \ -y_1 - 3y_2 - 3y_3 \leq -1\\
  &\qquad \qquad \ \ \ y_1, y_2, y_3 \leq 0
\end{align*}
(0, 0) is not a feasible point, so I create the auxillary problem:
\begin{align*}
&\text{maximize} \ \ -y_0 \\
  &\text{subject to} \ \ - 2y_1 - y_2 - 3y_3 -y_0 \leq -1 \\
  &\qquad \qquad \ \ \ -y_1 - 3y_2 - 3y_3 - y_0 \leq -1\\
  &\qquad \qquad \ \ \ y_1, y_2, y_3, y_0 \leq 0
\end{align*}
\begin{center}
  \def\arraystretch{1.2}
  \begin{tabular}{ C C C C C C C C C C C C C }
    \zeta & = & & & & & & & & - & v_0 \\
    \hline
    v_1 & = & -1 & + & 2y_1 & + & y_2 & + & 2y_3 & + & v_0 \\
    v_2 & = & -1 & + & y_1 & + & 3y_2 & + & 3y_3 & + & v_0 \\
    \hline \hline
    \zeta & = & -1 & + & 2y_1 & + & y_2 & + & 2y_3 & - & v_1 \\
    \hline
    v_0 & = & 1 & - & 2y_1 & - & y_2 & - & 2y_3 & + & v_1 \\
    v_2 & = & & - & y_1 & + & 2y_2 & + & y_3 & + & v_1 \\
    \hline \hline
    \zeta & = & & & & & & & & - & v_0 \\
    \hline
    y_2 & = & 1 & - & 2y_1 & - & 2y_3 & + & v_1 & - & v_0 \\
    v_2 & = & 2 & - & 5y_1 & - & 3y_3 & + & 3v_1 & - & 2v_0 \\
    \hline \hline
\end{tabular} 
\end{center}
Now I substitute into the objective function and solve the original problem.
\begin{center}
  \def\arraystretch{1.2}
  \begin{tabular}{ C C C C C C C C C C C C C }
    \zeta & = & -2 & + & y_1 & - & 3y_2 & - & 2v_1 \\
    \hline
    y_3 & = & \tfrac{1}{2} & - & y_1 & - & \tfrac{1}{2}y_2 & + & \tfrac{1}{2}v_1 \\
    v_2 & = & \tfrac{1}{2} & - & 2y_1 & + & \tfrac{3}{2}y_2 & + & \tfrac{3}{2}v_1 \\
    \hline \hline
    \zeta & = & -\tfrac{7}{4} & - & \tfrac{3}{2}y_2 & - & \tfrac{5}{4}v_1 & - & \tfrac{1}{2}v_2 \\
    \hline
    y_3 & = & \tfrac{1}{4} & - & 2\tfrac{3}{2}y_2 & - & \tfrac{1}{4}v_1 & + & \tfrac{1}{2}v_2 \\
    y_1 & = & \tfrac{1}{4} & + & \tfrac{3}{2}y_2 & + & \tfrac{3}{4}v_1 & - & \tfrac{1}{2}v_2 \\
    \hline
  \end{tabular}
\end{center}
The function attains its optimum $\tfrac{7}{4}$ at the point $(\tfrac{1}{4}, 0, \tfrac{1}{4})$ \\
The two problems match up which is good!

\end{problem}

\end{document}
